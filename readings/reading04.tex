\documentclass{homework}
\usepackage{amsmath}
\DeclareMathOperator{\Mat}{Mat}
\DeclareMathOperator{\End}{End}
\DeclareMathOperator{\Hom}{Hom}
\DeclareMathOperator{\id}{id}
\DeclareMathOperator{\image}{im}
\DeclareMathOperator{\rank}{rank}
\DeclareMathOperator{\nullity}{nullity}
\DeclareMathOperator{\trace}{tr}
\DeclareMathOperator{\Spec}{Spec}
\DeclareMathOperator{\Sym}{Sym}
\DeclareMathOperator{\pf}{pf}
\DeclareMathOperator{\Ortho}{O}
\DeclareMathOperator{\diam}{diam}
\DeclareMathOperator{\Real}{Re}
\DeclareMathOperator{\Imag}{Im}
\DeclareMathOperator{\Arg}{Arg}
\DeclareMathOperator{\Log}{Log}

\newcommand{\C}{\mathbb{C}}
\newcommand{\R}{\mathbb{R}}
\newcommand{\Z}{\mathbb{Z}}
\newcommand{\N}{\mathbb{N}}


\DeclareMathOperator{\sla}{\mathfrak{sl}}
\newcommand{\norm}[1]{\left\lVert#1\right\rVert}
\newcommand{\transpose}{\intercal}

\newcommand{\conj}[1]{\overline{#1}}
\newcommand{\abs}[1]{\left|#1\right|}

%%% My commands, for solutions %%%

\usepackage{amssymb}
\usepackage{xifthen}
\usepackage{listings}
\usepackage{tikz} % Guide http://bit.ly/gNfVn9
\usetikzlibrary{decorations.markings}
\DeclareMathOperator{\Res}{Res}

% To write df/(dx), use \pfrac{f}{x}
\newcommand{\pfrac}[2]{\frac{\partial #1}{\partial #2}}
% Partial derivative. To take d^2f/(dxdy), use \ppfrac[y]{f}{x}
% To take d^2f/(dx^2), use ppfrac{f}{x}
\newcommand{\ppfrac}[3][]{\frac{\partial^2 #2}{\ifthenelse{\isempty{#1}}{\partial #3^2}{\partial #3\partial #1}}}
\newcommand{\oo}[0]{\infty}

 \newenvironment{solution}
   {\renewcommand\qedsymbol{$\blacksquare$}\begin{proof}[Solution]}
     {\end{proof}}
       
       % Code listing environment  
       \lstnewenvironment{code}{\lstset{basicstyle=\ttfamily, mathescape=true, breaklines=true}}{}

       % When you want to see how many pages your HW is
       \usepackage{lastpage}
       \usepackage{fancyhdr}
       \pagestyle{fancy} 
       \cfoot{\thepage\ of \pageref{LastPage}}



\usepackage{hyperref}
\author{Jim Fowler}
\course{Math 5522H}
\title{Assigned Readings}
\date{Week 4}

\begin{document}
\maketitle

Having considered holomorphic functions for our first couple weeks
together, and having considered line integrals last week, it is time
to put these ideas together.  What is the integral of a holomoprhic
function around a closed curve?  \textit{``By Cauchy, it vanishes!''}
To review this, look to the beginning of Chapter V of Palka's
\textit{An Introduction to Complex Function Theory}, namely
\begin{itemize}
\item V.1.1 Cauchy's Theorem For Rectangles
\item V.1.2 Integrals and Primitives
\item V.1.3 The Local Cauchy Theorem
\end{itemize}
Alternatively from Ahlfors' \textit{Complex Analysis}, read
\begin{itemize}
\item 4.1.4 Cauchy's Theorem for a Rectangle
\item 4.1.5 Cauchy's Theorem in a Disk
\end{itemize}
or from Stein and Shakarchi's \textit{Complex Analysis}, read
\begin{itemize}
\item 2.1 Goursat's theorem
\item 2.2 Local existence of primitives and Cauchy's theorem in a disc
\end{itemize}
Pay attention to Goursat's theorem, which we have already seen.

Besides Cauchy's theorem, the other goal this week is to see some of
the major applications of Cauchy's theorem.  Read farther into Chapter
V of Palka's \textit{An Introduction to Complex Function Theory},
paying attention to
\begin{itemize}
\item V.2.1 Winding Numbers
\item V.2.2 Oriented Paths, Jordan Contours
\item V.2.3 The Local Integral Formula
\item V.3.1 Analyticity of Derivatives
\item V.3.2 Derivative Estimates
\item V.3.3 The Maximum Principle
\end{itemize}
Over the next couple weeks, we'll unwrap these and additional
consequences of the ``Cauchy-Goursat'' theorem and discover that
holomorphic functions are remarkably rigid objects.

\end{document}
