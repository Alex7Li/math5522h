\documentclass{homework}
\usepackage{amsmath}
\DeclareMathOperator{\Mat}{Mat}
\DeclareMathOperator{\End}{End}
\DeclareMathOperator{\Hom}{Hom}
\DeclareMathOperator{\id}{id}
\DeclareMathOperator{\image}{im}
\DeclareMathOperator{\rank}{rank}
\DeclareMathOperator{\nullity}{nullity}
\DeclareMathOperator{\trace}{tr}
\DeclareMathOperator{\Spec}{Spec}
\DeclareMathOperator{\Sym}{Sym}
\DeclareMathOperator{\pf}{pf}
\DeclareMathOperator{\Ortho}{O}
\DeclareMathOperator{\diam}{diam}
\DeclareMathOperator{\Real}{Re}
\DeclareMathOperator{\Imag}{Im}
\DeclareMathOperator{\Arg}{Arg}
\DeclareMathOperator{\Log}{Log}

\newcommand{\C}{\mathbb{C}}
\newcommand{\R}{\mathbb{R}}
\newcommand{\Z}{\mathbb{Z}}
\newcommand{\N}{\mathbb{N}}


\DeclareMathOperator{\sla}{\mathfrak{sl}}
\newcommand{\norm}[1]{\left\lVert#1\right\rVert}
\newcommand{\transpose}{\intercal}

\newcommand{\conj}[1]{\overline{#1}}
\newcommand{\abs}[1]{\left|#1\right|}

%%% My commands, for solutions %%%

\usepackage{amssymb}
\usepackage{xifthen}
\usepackage{listings}
\usepackage{tikz} % Guide http://bit.ly/gNfVn9
\usetikzlibrary{decorations.markings}
\DeclareMathOperator{\Res}{Res}

% To write df/(dx), use \pfrac{f}{x}
\newcommand{\pfrac}[2]{\frac{\partial #1}{\partial #2}}
% Partial derivative. To take d^2f/(dxdy), use \ppfrac[y]{f}{x}
% To take d^2f/(dx^2), use ppfrac{f}{x}
\newcommand{\ppfrac}[3][]{\frac{\partial^2 #2}{\ifthenelse{\isempty{#1}}{\partial #3^2}{\partial #3\partial #1}}}
\newcommand{\oo}[0]{\infty}

 \newenvironment{solution}
   {\renewcommand\qedsymbol{$\blacksquare$}\begin{proof}[Solution]}
     {\end{proof}}
       
       % Code listing environment  
       \lstnewenvironment{code}{\lstset{basicstyle=\ttfamily, mathescape=true, breaklines=true}}{}

       % When you want to see how many pages your HW is
       \usepackage{lastpage}
       \usepackage{fancyhdr}
       \pagestyle{fancy} 
       \cfoot{\thepage\ of \pageref{LastPage}}



\usepackage{hyperref}
\author{Jim Fowler}
\course{Math 5522H}
\title{Assigned Readings}
\date{Week 8}

\usepackage{draftwatermark}
\SetWatermarkText{Draft}
\SetWatermarkScale{5}

\begin{document}
\maketitle

% Singularities

From Palka's \textit{An Introduction to Complex Function Theory}, read
\begin{itemize}
\item VIII.1.1 The Factor Theorem for Analytic Functions
\item VIII.1.2 Multiplicity
\item VIII.1.3 Discrete Sets, Discrete Mappings
\item VIII.2.1 Definition and Classification of Isolated Singularities 309
\item VIII.2.2 Removable Singularities
\item VIII.2.3 Poles
\item VIII.2.4 Meromorphic Functions
\item VIII.2.5 Essential Singularities
\item VIII.2.6 Isolated Singularities at Infinity
\item VIII.4.1 The Extended Complex Plane
\item VIII.4.2 The Extended Plane and Stereographic Projection
\item VIII.4.3 Functions in the Extended Setting
\item VIII.4.4 Topology in the Extended Plane
\item VIII.4.5 Meromorphic Functions and the Extended Plane
\end{itemize}

From Stein and Shakarchi's \textit{Complex Analysis}, read
\begin{itemize}
\item 3.1 Zeros and poles
\item 3.3 Singularities and meromorphic functions
\end{itemize}

From Ahlfors' \textit{Complex Analysis: An Introduction to The Theory of Analytic Functions of One Complex Variable}, read
\begin{itemize}
\item 4.3 Local Properties of Analytical Functions
\item 4.3.1 Removable Singularities. Taylor's Theorem
\item 4.3.2 Zeros and Poles
\item 4.3.3 The Local Mapping
\item 4.3.4 The Maximum Principle
\end{itemize}
\end{document}
