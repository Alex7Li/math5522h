\documentclass{homework}
\usepackage{amsmath}
\DeclareMathOperator{\Mat}{Mat}
\DeclareMathOperator{\End}{End}
\DeclareMathOperator{\Hom}{Hom}
\DeclareMathOperator{\id}{id}
\DeclareMathOperator{\image}{im}
\DeclareMathOperator{\rank}{rank}
\DeclareMathOperator{\nullity}{nullity}
\DeclareMathOperator{\trace}{tr}
\DeclareMathOperator{\Spec}{Spec}
\DeclareMathOperator{\Sym}{Sym}
\DeclareMathOperator{\pf}{pf}
\DeclareMathOperator{\Ortho}{O}
\DeclareMathOperator{\diam}{diam}
\DeclareMathOperator{\Real}{Re}
\DeclareMathOperator{\Imag}{Im}
\DeclareMathOperator{\Arg}{Arg}
\DeclareMathOperator{\Log}{Log}

\newcommand{\C}{\mathbb{C}}
\newcommand{\R}{\mathbb{R}}
\newcommand{\Z}{\mathbb{Z}}
\newcommand{\N}{\mathbb{N}}


\DeclareMathOperator{\sla}{\mathfrak{sl}}
\newcommand{\norm}[1]{\left\lVert#1\right\rVert}
\newcommand{\transpose}{\intercal}

\newcommand{\conj}[1]{\overline{#1}}
\newcommand{\abs}[1]{\left|#1\right|}

%%% My commands, for solutions %%%

\usepackage{amssymb}
\usepackage{xifthen}
\usepackage{listings}
\usepackage{tikz} % Guide http://bit.ly/gNfVn9
\usetikzlibrary{decorations.markings}
\DeclareMathOperator{\Res}{Res}

% To write df/(dx), use \pfrac{f}{x}
\newcommand{\pfrac}[2]{\frac{\partial #1}{\partial #2}}
% Partial derivative. To take d^2f/(dxdy), use \ppfrac[y]{f}{x}
% To take d^2f/(dx^2), use ppfrac{f}{x}
\newcommand{\ppfrac}[3][]{\frac{\partial^2 #2}{\ifthenelse{\isempty{#1}}{\partial #3^2}{\partial #3\partial #1}}}
\newcommand{\oo}[0]{\infty}

 \newenvironment{solution}
   {\renewcommand\qedsymbol{$\blacksquare$}\begin{proof}[Solution]}
     {\end{proof}}
       
       % Code listing environment  
       \lstnewenvironment{code}{\lstset{basicstyle=\ttfamily, mathescape=true, breaklines=true}}{}

       % When you want to see how many pages your HW is
       \usepackage{lastpage}
       \usepackage{fancyhdr}
       \pagestyle{fancy} 
       \cfoot{\thepage\ of \pageref{LastPage}}



\usepackage{hyperref}
\author{Jim Fowler}
\course{Math 5522H}
\title{Assigned Readings}
\date{Week 2}

\usepackage{draftwatermark}
\SetWatermarkText{Draft}
\SetWatermarkScale{5}

\begin{document}
\maketitle

Last week, we met the complex numbers.  This week, we take a closer
look at the \textit{real} (!) star of complex analysis---the
\textbf{holomorphic functions.}  Holomorphic functions are those which
are complex differentiable, but complex differentiable is a much
stronger condition than you might guess it to be, and is related to
the \textbf{Cauchy-Riemann equations} we have already met.  Many of
our favorite functions from real analysis can be extended to
\textbf{entire} functions in the complex plane.

Early this week, we will see some consequences of Cauchy-Riemann
equations --- these will be our first glimpse of just how strong a
condition holomorphicity is.  We will be careful to distinguish, say,
functions which are differentiable when regarded on $\mathbb{R}^2$,
and functions which are complex differentiable, i.e., holomorphic.
When I say \textbf{smooth}, for instance, I mean functions in
$C^\infty(\mathbb{R}^2)$.

At the end of this week, we will finally meet \textbf{analytic
  functions}, i.e., we will review power series and see that many of
the results you remember from real analysis carry over to the complex
setting.  But the real surprise is yet to come: \textbf{holomorphic
  functions are analytic}.

It will often feel that I'm being pedantic with this incessent review
of (real) differentiation, but I hope by repeatedly reviewing
derivatives from a variety of perspectives, I will have indirectly
provided some intuition around tangent spaces --- and this intuition
will be helpful when we see 1-forms next week and plunge into line
integrals!

So open up Chapter III of Palka's \textit{An Introduction to Complex
  Function Theory} and peruse
\begin{itemize}
\item III.1.1 Differentiability
\item III.1.2 Differentiation Rules
\item III.1.3 Analytic Functions
\item III.2.1 The Cauchy-Riemann System of Equations
\item III.2.2 Consequences of the Cauchy-Riemann Relations
\item III.3.1 Entire Functions
\item III.3.2 Trigonometric Functions
\item III.3.3 The Principal Arcsine and Arctangent Functions
\item III.4.1 Branches of Inverse Functions
\item III.4.2 Branches of the $p^{th}$-root Function
\item III.4.3 Branches of the Logarithm Function
\item III.4.4 Branches of the power Function
\item III.5.1 Real Differentiability
\item III.5.2 The Functions $f_z$ and $f_{\bar{z}}$
\end{itemize}
Alternatively, you can see some similar content in Ahlfors' \textit{Complex Analysis}, namely
\begin{itemize}
\item 2.1 Introduction to the Concept of Analytic Function
\item 2.1.1 Limits and Continuity
\item 2.1.2 Analytic Functions
\item 2.1.3 Polynomials
\item 2.1.4 Rational Functions
\item 2.2 Elementary Theory of Power Series
\item 2.2.1 Sequences
\item 2.2.2 Series
\item 2.2.3 Uniform Coverages
\item 2.2.4 Power Series
\item 2.2.5 Abel's Limit Theorem
\end{itemize}
Or you can read Stein and Shakarchi's \textit{Complex Analysis} by focusing on
\begin{itemize}
\item 1.2 Functions on the complex plane
\item 1.2.1 Continuous functions
\item 1.2.2 Holomorphic functions
\item 1.2.3 Power series
\end{itemize}


\end{document}
