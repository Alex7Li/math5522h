\documentclass{homework}
\usepackage{amsmath}
\DeclareMathOperator{\Mat}{Mat}
\DeclareMathOperator{\End}{End}
\DeclareMathOperator{\Hom}{Hom}
\DeclareMathOperator{\id}{id}
\DeclareMathOperator{\image}{im}
\DeclareMathOperator{\rank}{rank}
\DeclareMathOperator{\nullity}{nullity}
\DeclareMathOperator{\trace}{tr}
\DeclareMathOperator{\Spec}{Spec}
\DeclareMathOperator{\Sym}{Sym}
\DeclareMathOperator{\pf}{pf}
\DeclareMathOperator{\Ortho}{O}
\DeclareMathOperator{\diam}{diam}
\DeclareMathOperator{\Real}{Re}
\DeclareMathOperator{\Imag}{Im}
\DeclareMathOperator{\Arg}{Arg}
\DeclareMathOperator{\Log}{Log}

\newcommand{\C}{\mathbb{C}}
\newcommand{\R}{\mathbb{R}}
\newcommand{\Z}{\mathbb{Z}}
\newcommand{\N}{\mathbb{N}}


\DeclareMathOperator{\sla}{\mathfrak{sl}}
\newcommand{\norm}[1]{\left\lVert#1\right\rVert}
\newcommand{\transpose}{\intercal}

\newcommand{\conj}[1]{\overline{#1}}
\newcommand{\abs}[1]{\left|#1\right|}

%%% My commands, for solutions %%%

\usepackage{amssymb}
\usepackage{xifthen}
\usepackage{listings}
\usepackage{tikz} % Guide http://bit.ly/gNfVn9
\usetikzlibrary{decorations.markings}
\DeclareMathOperator{\Res}{Res}

% To write df/(dx), use \pfrac{f}{x}
\newcommand{\pfrac}[2]{\frac{\partial #1}{\partial #2}}
% Partial derivative. To take d^2f/(dxdy), use \ppfrac[y]{f}{x}
% To take d^2f/(dx^2), use ppfrac{f}{x}
\newcommand{\ppfrac}[3][]{\frac{\partial^2 #2}{\ifthenelse{\isempty{#1}}{\partial #3^2}{\partial #3\partial #1}}}
\newcommand{\oo}[0]{\infty}

 \newenvironment{solution}
   {\renewcommand\qedsymbol{$\blacksquare$}\begin{proof}[Solution]}
     {\end{proof}}
       
       % Code listing environment  
       \lstnewenvironment{code}{\lstset{basicstyle=\ttfamily, mathescape=true, breaklines=true}}{}

       % When you want to see how many pages your HW is
       \usepackage{lastpage}
       \usepackage{fancyhdr}
       \pagestyle{fancy} 
       \cfoot{\thepage\ of \pageref{LastPage}}



\usepackage{hyperref}
\author{Jim Fowler}
\course{Math 5522H}
\title{Assigned Readings}
\date{Week 1}

\usepackage{draftwatermark}
\SetWatermarkText{Draft}
\SetWatermarkScale{5}

\begin{document}
\maketitle

I've heard complex analysis described as the Disneyland of mathematics
--- as Disneyland is ``the happiest place on Earth,'' perhaps complex
analysis is the ``happiest place in mathematics'' where the theorems
we always wished were true finally are true.  Polynomials have roots,
functions are infinitely differentiable, challenging integrals can be
computed!

But our first task is to gain admission into Disneyland.  So our goal
for our first week, in addition to getting used to the format of the
course, is to meet the \textbf{complex numbers} $\mathbb{C}$.  Complex
numbers appear in the
\href{http://www.corestandards.org/Math/Content/HSN/CN/}{Common Core
  State Standards}, so I assume you first met the complex numbers back
in high school.  Nevertheless, as is traditional, a course in complex
analysis begins by reviewing a certain amount of complex arithmetic,
starting with the key \textbf{terminology} question on Problem Set 1:
just what do we mean, precisely, by $\mathbb{C}$, the complex numbers?

With that out of the way, the \textbf{numericals} on Problem Set 1
will provide our first practice with such arithmetic with fun examples
like \textbf{partial fractions} and \textbf{Gauss sums}.

To dig into these ideas, read Palka's \textit{An Introduction to
  Complex Function Theory}, specifically paying attention to
\begin{itemize}
\item I.1.1 The Field of Complex Numbers,
\item I.1.2 Conjugate, Modulus, and Argument,
\item I.2.1 Raising $e$ to Complex Powers,
\item I.2.2 Logarithms of Complex Numbers,
\item I.2.3 Raising Complex Numbers to Complex Powers.
\end{itemize}
Another reference is Lars Ahlfors' classic text \textit{Complex
  Analysis: An Introduction to The Theory of Analytic Functions of One
  Complex Variable}.  (Perhaps the main challenge with that text is
that the theorems, woven into the narrative, are not always so clearly
called out or labeled.)  In Ahlfors' book, you could consult
Section 1.1 The Algebra of Complex Numbers and Section 1.2 The Geometric Representation of Complex Numbers, namely
\begin{itemize}
\item 1.1.1 Arithmetic Operations,
\item 1.1.2 Square Roots,
\item 1.1.3 Justification,
\item 1.1.4 Conjugation, Absolute Value,
\item 1.1.5 Inequalities,
\item 1.2.1 Geometric Addition and Multiplication,
\item 1.2.2 The Binomial Equation,
\item 1.2.3 Analytic Geometry,
\item 1.2.4 The Spherical Representation.
\end{itemize}
A third reference for this course is the somewhat more recently
published \textit{Complex Analysis} by Stein and Shakarchi, which
launches quite a bit more quickly.  Although Palka will serve as our
main text, let's try to keep up with the brisker pace of Stein and
Shakarchi.  In their text, take a look at the short sections
\begin{itemize}
\item 1.1 Complex numbers and the complex plane
\item 1.1.1 Basic properties
\item 1.1.2 Convergence
\item 1.1.3 Sets in the complex plane
\end{itemize}
You may worry that we are \textit{not} formally covering Chapter II of
Palka; as Palka recommends in the Preface, we will review the
topological preliminaries following a ``just-in-time'' approach,
recalling those ideas as needed.  Problem Set 1 includes problems with
invite you to make connections with linear algebra (e.g., M\"obius
transformations) and review some real analysis (e.g., completeness,
Abel's theorem, nested subsets of the plane).

\textbf{Next week} we will focus on \textit{functions} of a complex
variable, but we can get a preview by consulting Palka's \textit{An
  Introduction to Complex Function Theory} in Section I.3 Functions of
a Complex Variable, namely
\begin{itemize}
\item I.3.1 Complex Functions
\item I.3.2 Combining Functions
\item I.3.3 Functions as Mappings
\end{itemize}
Problem Set 1 already introduces us to some important examples, like
\textbf{Blaschke factors} and the \textbf{cross-ratio}, and also
introduces the important class of \textbf{harmonic functions}.

\end{document}
