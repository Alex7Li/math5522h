\documentclass{homework}
\course{Math 5522H}
\author{Jim Fowler}
\usepackage{amsmath}
\DeclareMathOperator{\Mat}{Mat}
\DeclareMathOperator{\End}{End}
\DeclareMathOperator{\Hom}{Hom}
\DeclareMathOperator{\id}{id}
\DeclareMathOperator{\image}{im}
\DeclareMathOperator{\rank}{rank}
\DeclareMathOperator{\nullity}{nullity}
\DeclareMathOperator{\trace}{tr}
\DeclareMathOperator{\Spec}{Spec}
\DeclareMathOperator{\Sym}{Sym}
\DeclareMathOperator{\pf}{pf}
\DeclareMathOperator{\Ortho}{O}
\DeclareMathOperator{\diam}{diam}
\DeclareMathOperator{\Real}{Re}
\DeclareMathOperator{\Imag}{Im}
\DeclareMathOperator{\Arg}{Arg}
\DeclareMathOperator{\Log}{Log}

\newcommand{\C}{\mathbb{C}}
\newcommand{\R}{\mathbb{R}}
\newcommand{\Z}{\mathbb{Z}}
\newcommand{\N}{\mathbb{N}}


\DeclareMathOperator{\sla}{\mathfrak{sl}}
\newcommand{\norm}[1]{\left\lVert#1\right\rVert}
\newcommand{\transpose}{\intercal}

\newcommand{\conj}[1]{\overline{#1}}
\newcommand{\abs}[1]{\left|#1\right|}

%%% My commands, for solutions %%%

\usepackage{amssymb}
\usepackage{xifthen}
\usepackage{listings}
\usepackage{tikz} % Guide http://bit.ly/gNfVn9
\usetikzlibrary{decorations.markings}
\DeclareMathOperator{\Res}{Res}

% To write df/(dx), use \pfrac{f}{x}
\newcommand{\pfrac}[2]{\frac{\partial #1}{\partial #2}}
% Partial derivative. To take d^2f/(dxdy), use \ppfrac[y]{f}{x}
% To take d^2f/(dx^2), use ppfrac{f}{x}
\newcommand{\ppfrac}[3][]{\frac{\partial^2 #2}{\ifthenelse{\isempty{#1}}{\partial #3^2}{\partial #3\partial #1}}}
\newcommand{\oo}[0]{\infty}

 \newenvironment{solution}
   {\renewcommand\qedsymbol{$\blacksquare$}\begin{proof}[Solution]}
     {\end{proof}}
       
       % Code listing environment  
       \lstnewenvironment{code}{\lstset{basicstyle=\ttfamily, mathescape=true, breaklines=true}}{}

       % When you want to see how many pages your HW is
       \usepackage{lastpage}
       \usepackage{fancyhdr}
       \pagestyle{fancy} 
       \cfoot{\thepage\ of \pageref{LastPage}}




\begin{document}
\maketitle

\begin{inspiration}
Some of the most important results (e.g. Cauchy's theorem) are so surprising at first sight that nothing short of a proof can make them credible.
\byline{Sir Harold Jeffreys} % and where is this quotation from?
\end{inspiration}

\section{Terminology}

\begin{problem}
  What is a \textbf{Jordan curve}?
\end{problem}

\begin{problem}
  Define the \textbf{winding number} $n(\gamma,z)$ of a piecewise
  smooth closed curve $\gamma$ around a point $z \in \C$.
\end{problem}

\section{Numericals}

\begin{problem} Compute the \textbf{Fresnel integrals}
  \[
    \int_0^\infty \sin \left( x^2 \right) \, dx \mbox{ and } \int_0^\infty \cos \left( x^2 \right) \, dx.
  \]
\end{problem}

\begin{problem}
  For an integrable function $f : \R \to \C$, define the
  \textbf{Fourier transform} of $f$, denoted $\hat{f}$, by
  \[
    {\hat {f}}(\xi ) := \int _{-\infty }^{\infty} f(x) \, e^{-2\pi ix \xi} \,dx.
  \]
  Find the Fourier transform of $f(x) = e^{-\pi x^2}$.
\end{problem}

\section{Exploration}

\begin{problem}
  Does the function $f : \C \to \C$ given by $f(z) = \conj{z}$ have a
  primitive?
\end{problem}

\begin{problem}
  We proved Goursat's theorem for rectangles.  Without simply
  recapitalulating the proof (e.g., your argument should not again
  invoke \ref{nested-subsets-convergence}), deduce a theorem for
  \textit{triangles} from our result about rectangles.
\end{problem}

\begin{problem}
  Suppose $f, g : \C \to \C$ are holomorphic and agree on the unit
  circle.  How do $f$ and $g$ relate?  Morally, this problem is
  related to \ref{identity-theorem}.
\end{problem}

 \begin{problem}
   Suppose $f : \C \to \C$ is holomorphic, and consider the circle
   $\gamma$ with center $z_0$.  How does the real part of the average
   value of $f$ on the circle $\gamma$ relate to $f(z_0)$?
 \end{problem}

 \begin{problem}
   Recalling \ref{harmonic-conjugate}, suppose $u, v : \C \to \R$ are
   harmonic functions and $f = u + iv$ is holomorphic, so $u$ and $v$
   are harmonic conjugates.  Find the \textbf{Poisson kernel} for the
   unit disc, i.e., find a function $P_r(\theta)$ so that
   \[
     u(re^{i\theta}) = \frac {1}{2\pi} \int_{-\pi }^{\pi } P_{r}(\theta -t) \, u(e^{it}) \, dt
   \]
   for $r < 1$.
 \end{problem}

\section{Prove or Disprove and Salvage if Possible}

\begin{problem}
  Suppose $D = \{ z \in \C : \abs{z} < 1 \}$ and $f : D \to \C$ is
  continuous and $\gamma : [a,b] \to D$ is a piecewise smooth curve.
  Then $\displaystyle\int_\gamma f\, dz = 0$.
\end{problem}
  
\begin{problem}
  Suppose $U$ is an open set and $f : U \to \C$ is analytic and
  $\gamma : [a,b] \to U$ is a piecewise smooth curve.  Then
  $\displaystyle\int_\gamma f \, dz = 0$.
\end{problem}

\begin{problem}
  Define $\gamma_r : [0,2\pi] \to \C$ by
  $\gamma_r(\theta) = r e^{i \theta}$, and for real numbers
  $R > r > 0$, define the annulus
  \[
    A(r,R) := \{ z \in \C : r < |z| < R \}
  \]
  and suppose $f : A(r/2,2R) \to \C$ is holomorphic.  Then
  \[
    \int_{\gamma_r} f \, dz = \int_{\gamma_R} f \, dz.
  \]
\end{problem}

\begin{problem}\label{cauchy-integral-formula}Define $B_R(0) := \{ z \in C : \abs{z} < R \}$ and suppose
  $f : B_R(0) \to \C$ is holomorphic and let $\gamma$ be the
  positively-oriented circle of positive radius $r < R$.  Then
  \[
    f(a) = \int_\gamma \frac{f(z)}{z-a} \, dz.
  \]
  and moreover
  \[
    f'(a) = \int_\gamma \frac{f(z)}{(z-a)^2} \, dz.
  \]  
\end{problem}

\begin{problem}\label{cauchy-inequalities}If $f : U \to \C$ is
  holomorphic and $U \supset B_r(z_0)$, then
  \[
     \abs{f(z_0)} \leq \sup_{z \in \partial B_r(z_0)} \abs{f(z)}
   \]
   and
   \[
     \abs{f'(z_0)} \leq \sup_{z \in \partial B_r(z_0)} \abs{f(z)}. % missing (1/r) factor
   \]
 \end{problem}

\begin{problem}\label{liouville-theorem}Recall that a function $f : U \to \C$ is
  \textbf{bounded} if there exists $M > 0$ so that for all $z \in U$
  we have $\abs{f(z)} \leq M$.  A bounded holomorphic function
  $f : U \to \C$ is constant.
\end{problem}

\begin{problem}
  For every polynomial $p \in \C[z]$ there is $z \in \C$ so that $p(z) = 0$.
\end{problem}

\end{document}
