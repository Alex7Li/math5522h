\documentclass{homework}
\course{Math 5522H}
\author{Jim Fowler}
\usepackage{amsmath}
\DeclareMathOperator{\Mat}{Mat}
\DeclareMathOperator{\End}{End}
\DeclareMathOperator{\Hom}{Hom}
\DeclareMathOperator{\id}{id}
\DeclareMathOperator{\image}{im}
\DeclareMathOperator{\rank}{rank}
\DeclareMathOperator{\nullity}{nullity}
\DeclareMathOperator{\trace}{tr}
\DeclareMathOperator{\Spec}{Spec}
\DeclareMathOperator{\Sym}{Sym}
\DeclareMathOperator{\pf}{pf}
\DeclareMathOperator{\Ortho}{O}
\DeclareMathOperator{\diam}{diam}
\DeclareMathOperator{\Real}{Re}
\DeclareMathOperator{\Imag}{Im}
\DeclareMathOperator{\Arg}{Arg}
\DeclareMathOperator{\Log}{Log}

\newcommand{\C}{\mathbb{C}}
\newcommand{\R}{\mathbb{R}}
\newcommand{\Z}{\mathbb{Z}}
\newcommand{\N}{\mathbb{N}}


\DeclareMathOperator{\sla}{\mathfrak{sl}}
\newcommand{\norm}[1]{\left\lVert#1\right\rVert}
\newcommand{\transpose}{\intercal}

\newcommand{\conj}[1]{\overline{#1}}
\newcommand{\abs}[1]{\left|#1\right|}

%%% My commands, for solutions %%%

\usepackage{amssymb}
\usepackage{xifthen}
\usepackage{listings}
\usepackage{tikz} % Guide http://bit.ly/gNfVn9
\usetikzlibrary{decorations.markings}
\DeclareMathOperator{\Res}{Res}

% To write df/(dx), use \pfrac{f}{x}
\newcommand{\pfrac}[2]{\frac{\partial #1}{\partial #2}}
% Partial derivative. To take d^2f/(dxdy), use \ppfrac[y]{f}{x}
% To take d^2f/(dx^2), use ppfrac{f}{x}
\newcommand{\ppfrac}[3][]{\frac{\partial^2 #2}{\ifthenelse{\isempty{#1}}{\partial #3^2}{\partial #3\partial #1}}}
\newcommand{\oo}[0]{\infty}

 \newenvironment{solution}
   {\renewcommand\qedsymbol{$\blacksquare$}\begin{proof}[Solution]}
     {\end{proof}}
       
       % Code listing environment  
       \lstnewenvironment{code}{\lstset{basicstyle=\ttfamily, mathescape=true, breaklines=true}}{}

       % When you want to see how many pages your HW is
       \usepackage{lastpage}
       \usepackage{fancyhdr}
       \pagestyle{fancy} 
       \cfoot{\thepage\ of \pageref{LastPage}}




\begin{document}
\maketitle

\begin{inspiration}
Le plus court chemin entre deux v\'erit\'es dans le domaine r\'eel passe par le domaine complexe.
%The shortest path between two truths in the real domain passes through the complex domain.
\byline{Jacques Hadamard}
\end{inspiration}

\section{Terminology}

\begin{problem}
  Define $\C$.  (How many ``different'' definitions do you know?)
\end{problem}

\begin{problem}
  Define $\conj{z}$ and $\abs{z}$ for $z \in \C$.
\end{problem}

\begin{problem}
  For complex numbers $z, w \in \C$, what do we mean by $z^w$ ?
\end{problem}

\section{Numericals}

\begin{problem}
  Apply \textbf{partial fractions} to write $\displaystyle\frac{1}{1-z^4}$ as a sum of terms of the form $\displaystyle\frac{A}{Bz + C}$.
\end{problem}

\begin{problem}
  Find $a, b, z \in \C$ so that $\left(z^a\right)^b \neq z^{\left(ab\right)}$.
\end{problem}

\begin{problem}
  We will often see \textbf{roots of unity}.  To practice computing with such objects, let
  \[
    \zeta := \cos \left( \frac{2\pi}{7} \right) + i \, \sin \left( \frac{2\pi}{7} \right) \mbox{ and }
    r := \zeta + \zeta^2 - \zeta^3 + \zeta^4 - \zeta^5 - \zeta^6.
  \]
  Find the integer $r^2$.  (This surprise is a \textbf{Gauss sum}.)
\end{problem}

\begin{problem}
  For which $z \in \mathbb{C}$ is it the case that $\log \left( e^z \right) = z$?  \\ (What do we mean when we write $\log$ here?)
\end{problem}

\section{Exploration}

\begin{problem}
  Let's review some linear algebra.  Define $J(x,y) = (y,-x)$ so $J$
  is counter-clockwise rotation by $90^\circ$, and suppose
  $T : \R^2 \to \R^2$ is a linear transformation with the property
  that $T \circ J = J \circ T$.  Can you relate $T$ to the complex
  numbers?
\end{problem}

\begin{problem}\label{mobius-transformations}Here is another connection to linear algebra.  Suppose we have complex-valued functions
  \[
    f(z) = \frac{az + b}{cz + d} \mbox{ and }
    F(z) = \frac{Az + B}{Cz + D}.
  \]
  Such functions are \textbf{M\"obius transformations}.  Relate the
  function $f \circ F$ to a product of certain matrices.
\end{problem}

\begin{problem}\label{abels-theorem}Let's review some real analysis.  Consider a sequence $(a_n)$ of real number so that $\sum_{n=0}^\infty a_n$ converges to $L$.  Does the one-sided limit
  \[
    \lim_{x \to 1^{-}} \sum_{n=0}^\infty a_n x^n
  \]
  also equal $L$?  See \textbf{Abel's theorem}.
\end{problem}

\begin{problem}
  For an open subset $U \subset \R^2$, a \textbf{harmonic function} $f : U \to \R$ is a twice continuously differential function satisfying the Laplace's equation
  \[
    \frac{\partial^2 f}{\partial x^2} + \frac{\partial^2 f}{\partial y^2} = 0.
  \]
  Suppose $f(x,y) = Ax^3 + Bx^2 y + C xy^2 + D y^3$ is harmonic for constants $A, B, C, D \in \R$.  Relate $f$ to $z \cdot (x + iy)^3$.
\end{problem}

\section{Prove or Disprove and Salvage if Possible (PODASIP)}

\textit{You may not have met these PODASIP-style problems before.
  What follows are statements which may be ``true'' or ``false.''  If
  the statement is true, then you should provide a proof.  If the
  statement is false, find a counterexample and fix the statement and
  then supply a proof for your repaired statement.  For example,
  \ref{blaschke-factors} suffers from the domain of $f$ not being
  specified, so you can ``salvage'' \ref{blaschke-factors} by
  carefully describing the domain.}

\begin{problem}\label{blaschke-factors}
  Suppose $w \in \C$ and $\abs{w} < 1$.  Define a function by the rule
  \[
    f(z) = \frac{w - z}{1 - \conj{w}z}.
  \]
  If $\abs{z} < 1$, then $\abs{f(z)} < 1$.  (These are \textbf{Blaschke factors}.)
\end{problem}

\begin{problem} % you may assume R is complete.
  The field $\mathbb{C}$ is complete.
\end{problem}

\begin{problem}
 For all $z, w \in \C$ it is the case that $\sqrt{z} \sqrt{w} = \sqrt{zw}$.
\end{problem}

\begin{problem} % missing non-empty, compact
  Suppose $K_1 \supset K_2 \supset \cdots$ be nested subsets of $\C$ so that $\diam K_n < 1/2^n$.  Then
  \[
    \bigcap_{n=1}^\infty K_n
  \]
  is non-empty and consists of a single point.
\end{problem}

\begin{problem}
 For all $a, b, z \in \C$ it is the case that $z^a \, z^b = z^{a+b}$.
\end{problem}

\begin{problem}\label{cross-ratio}
  Define the \textbf{cross-ratio} of distinct complex numbers $z_1,z_2,z_3,z_4 \in \C$ by
\[
\left(z_1,z_2;z_3,z_4\right):=\frac {\left(z_3-z_1\right)\,\left(z_4-z_2\right)}{\left(z_3-z_2\right)\,\left(z_4-z_1\right)}.
 \]
 If $f$ is a M\"obius transformation (cf.~\ref{mobius-transformations}), then
 \[
   \left(z_1,z_2;z_3,z_4\right) =
   \left(f(z_1),f(z_2);f(z_3),f(z_4)\right).
\]
\end{problem}

\end{document}
