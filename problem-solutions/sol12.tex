\documentclass{homework}
\course{Math 5522H}
\author{Alex Li}
\usepackage{amsmath}
\DeclareMathOperator{\Mat}{Mat}
\DeclareMathOperator{\End}{End}
\DeclareMathOperator{\Hom}{Hom}
\DeclareMathOperator{\id}{id}
\DeclareMathOperator{\image}{im}
\DeclareMathOperator{\rank}{rank}
\DeclareMathOperator{\nullity}{nullity}
\DeclareMathOperator{\trace}{tr}
\DeclareMathOperator{\Spec}{Spec}
\DeclareMathOperator{\Sym}{Sym}
\DeclareMathOperator{\pf}{pf}
\DeclareMathOperator{\Ortho}{O}
\DeclareMathOperator{\diam}{diam}
\DeclareMathOperator{\Real}{Re}
\DeclareMathOperator{\Imag}{Im}
\DeclareMathOperator{\Arg}{Arg}
\DeclareMathOperator{\Log}{Log}

\newcommand{\C}{\mathbb{C}}
\newcommand{\R}{\mathbb{R}}
\newcommand{\Z}{\mathbb{Z}}
\newcommand{\N}{\mathbb{N}}


\DeclareMathOperator{\sla}{\mathfrak{sl}}
\newcommand{\norm}[1]{\left\lVert#1\right\rVert}
\newcommand{\transpose}{\intercal}

\newcommand{\conj}[1]{\overline{#1}}
\newcommand{\abs}[1]{\left|#1\right|}

%%% My commands, for solutions %%%

\usepackage{amssymb}
\usepackage{xifthen}
\usepackage{listings}
\usepackage{tikz} % Guide http://bit.ly/gNfVn9
\usetikzlibrary{decorations.markings}
\DeclareMathOperator{\Res}{Res}

% To write df/(dx), use \pfrac{f}{x}
\newcommand{\pfrac}[2]{\frac{\partial #1}{\partial #2}}
% Partial derivative. To take d^2f/(dxdy), use \ppfrac[y]{f}{x}
% To take d^2f/(dx^2), use ppfrac{f}{x}
\newcommand{\ppfrac}[3][]{\frac{\partial^2 #2}{\ifthenelse{\isempty{#1}}{\partial #3^2}{\partial #3\partial #1}}}
\newcommand{\oo}[0]{\infty}

 \newenvironment{solution}
   {\renewcommand\qedsymbol{$\blacksquare$}\begin{proof}[Solution]}
     {\end{proof}}
       
       % Code listing environment  
       \lstnewenvironment{code}{\lstset{basicstyle=\ttfamily, mathescape=true, breaklines=true}}{}

       % When you want to see how many pages your HW is
       \usepackage{lastpage}
       \usepackage{fancyhdr}
       \pagestyle{fancy} 
       \cfoot{\thepage\ of \pageref{LastPage}}




\begin{document}
\maketitle

\begin{inspiration}
What kind of metabolism do they have? What's their \textbf{growth rate?}
\byline{the fictional Dr. Alan Grant in \textit{Jurassic Park}, a well-known movie about complex dynamics}
\end{inspiration}

\textbf{This is a short week with an instructional break, so this
  problem set is shorter.}

  \section{Terminology}

  \begin{problem}
    What does it mean to say that an entire function is of
      \textbf{finite order}?
      \end{problem}
      \begin{solution}
      A function $f:\C\to\C$ is said to have order of growth at most $\rho\in\R$ if there exist $A\in\R, B\in\R$ so that $\forall z\, \in \C$,
      \[
      f(z) \leq Ae^{B\rho^{|z|}}
      \]
      The order is finite if such $\rho$ exists.
      \end{solution}
      \begin{problem}
        What are \textbf{canonical factors}?
        \end{problem}
        \begin{solution}
        For $k\geq0,$ the $k$th canonical factor is a function of $z$ given by the equation
        \[
        \left(1-z\right)e^{z + \frac{z^2}{2} + \dots + \frac{z^k}{k}}
        \]
        Note that when $k=0$, the exponential term is $e^0$.
        \end{solution}
        \section{Numericals}

        \begin{problem}
          Compute $\displaystyle\prod_{n=0}^\infty \left( 1 + z^{(2^n)} \right)$.
          \end{problem}
          \begin{solution}
          First we note that if $|z| < 1$, then $\sum_{n=0}^\oo |z^{(2^n)}| < \infty$ by comparison to the geometric series $z^n$, so the product converges to something nonzero, and if $|z| > 1$, then each term grows unboundedly so the product diverges. If $|z|=1$ then the product diverges (to 0) since no matter how big $n$ is, there will a $\delta$ so that there are terms of product farther than $\delta$ from $1$.

          Next let's consider this as a formal product and expand. We note that
          for any term $z^k$, there is eactly one way to multiple terms to get it, corresponding to the binary expansion of $k$:
          \[
          \prod_{n=0}^\infty \left( 1 + z^{(2^n)} \right) = \sum_{n=0}^\infty z^n
          \]
          The rearrangement of the terms of the product are justified by the fact that the sum $\sum_{n=0}^\infty z^n$ converges absolutely for $|z|< 1$. % Not really

          This implies that the sum is equal to 
          \[
          1 + \log(\frac{1}{1-z})
          \]
          \end{solution}
          \begin{problem}
            Compute $\displaystyle\prod_{n=1}^\infty \left( 1 - \frac{1}{(2n)^2} \right)$.
            \end{problem}
            \begin{solution}
            Recall that in an earlier assignment, we proved that 
            \[
            \sin \left( \pi z \right) = \pi z \prod_{n=1}^\infty \left( 1 - \frac{z^2}{n^2} \right)
            \]
            Then take $z=\frac{1}{2}$ to see that 
            \[
            \frac{2}{\pi} = \frac{2\sin \left( \pi/2 \right)}{\pi} = \prod_{n=1}^\infty \left( 1 - \frac{1}{(2n)^2} \right)
            \]
            \end{solution}
            \begin{problem}
              What is the order of growth of $f(z) = \sin z$?
              \end{problem}
              \begin{solution}
              \begin{align*}
              \abs{\sin z} = \abs{\frac{e^{iz} - e^{-iz}}{2}}
              \end{align*}
              For a radius $r$, both terms in the numerator are bounded above by $e^{\abs{z}}$, so 
              \begin{align*}
              \abs{\sin z} \leq e^{\abs{z}}
              \end{align*}
              Thus the order of growth of $f$ is at most $1$.

              Then considering the point $z=ir$ for $r>1$, 
              \[
              e^{iz} - e^{-iz} = e^{r} - e^{-r} \geq \frac{1}{2}e^{r}
              \]
              so the order of growth of $f$ is at least $1$. Hence the order of growth is exactly 1.
              \end{solution}
              \begin{problem}
                What is the order of growth of a polynomial?
                \end{problem}
                \begin{solution}
                For fixed $B, C, \rho>0$, the function $Ae^{B\rho^{|z|}}$ grows at an expoential rate in $|z|$, faster than any polynomial. Since $\rho\leq0$ will not satisfy any function, the inf of the $\rho$ satisfying the conditions is 0, so the order of growth of any polynomial is 0.
                \end{solution}
                \section{Exploration}

                \begin{problem}
                  Suppose $f : \C \to \C \setminus \{0, 1\}$ is a holomorphic function
                    of finite order.  What does Hadamard say about $f$ in this case?
                    \end{problem}
                    \begin{solution}
                    $f$ has no zeros, just like the function $g(z) = 1$. Hence by Hadamard's theorem, $f(z) = e^{p(z)}g(z)$. Since $f$ has finite order, $p(z)$ is a polynomial. If $p(z)$ is constant, than $f(z)$ is a constant. Otherwise, it has a root at $z_0$ by the funamental theorem of algebra. But $f(z_0) = e^p(z_0)g(z_0) = 1$, contradicting the range of $f$. Thus $f$ must be constant.
                    \end{solution}
                    \begin{problem}
                      Consider the function
                        \[
                            G(z) = z \prod_{n=1}^\infty \left( 1 + \frac{z}{n} \right) e^{-z/n}
                              \]
                                which has zeros at nonpositive integers, just like the function
                                  $z \mapsto 1/\Gamma(z)$ we saw last week.

                                    Is it the case that $1/\Gamma(z) = G(z)$?
                                    \end{problem}
                                    \begin{solution}
                                    Suppose that $G(z)\Gamma(z) = 1$. Then $\frac{G(2)}{G(1)} = \frac{\Gamma(2)}{\Gamma(1)} = 2$. But
                                    \begin{align*}
                                    \frac{G(2)}{G(1)} &= 2\prod_{n=1}^\infty\frac{1+\frac{2}{n}}{1 + \frac{1}{n}}e^{\frac{-1}{n}}\\
                                    &= 2\prod_{n=1}^\infty \frac{n + 2}{n + 1}e^{-\frac{1}{n}}\\
                                    \end{align*}
                                    This product telescopes, so
                                    \begin{align*}
                                    \frac{g(2)}{g(1)} &= \lim_{N\to\infty} N\exp({-\sum_{n=1}^N \frac{1}{n}})\\
                                    &= \lim_{N\to\infty} N\exp({-\sum_{n=1}^N \frac{1}{n}})e^{\log(N^{-1}) + \log(N))}\quad\color{purple} \text{ multiply by 1}\\
                                    &= \lim_{N\to\infty} Ne^{\log(N^{-1})}\exp({-\sum_{n=1}^N \frac{1}{n} + \log(N)})\\
                                    &= e^{-\gamma} \quad\color{purple} \gamma \text{ the euler mascheroni constant} \neq 2
                                    \end{align*}
                                    Thus $\frac{1}{\Gamma(z)\neq G(z)}$. One can also note that, since $\Gamma$ has order 1 and shares the same zeroes as $G$, $\Gamma(z) e^{a+bz} = G(z)$. We showed $b=-\gamma$.
                                    \end{solution}

                                    \section{Prove or Disprove and Salvage if Possible}

                                    \begin{problem}
                                      A meromorphic function on the complex plane is the quotient of two
                                        entire functions.
                                        \end{problem}
                                        \begin{proof}
                                        The claim is true. Let $f$ be meromorphic. If $f=0$, then $f(z)=g(z)/h(z)$ with $g(z)=0$, $h(z)=1$. Otherwise, $f$ has a list of poles with repetition $z_1, z_2, \dots$. Then there is an entire function $h$ with zeros at exactly the poles of $f$, given by a Wierstrauss infinite product. The function $h\cdot f = g$ then has removable singularities at $z_1, z_2, \dots$, and we can regard it as entire. Dividing by $h$, we see that $f = \frac{g}{h}$, so $f$ is holomorphic.
                                        \end{proof}
                                        \begin{problem}
                                          If $f : \C \to \C$ is a odd function of order 1 with $f(n) = 0$ for
                                            $n \in \Z$, then $f(z) = \sin (\pi z)$.
                                            \end{problem}
                                            \begin{solution}
                                            False. $f(z) = z^2\sin (\pi z)$ also works.


                                            If we assume that $f$ has a zero of degree 1 at 0, we can get the result up to a constant:

                                            By Hadamard's factorization theorem, if $f$ is of order 1, then we can write it as a product of $\sin(\pi z)$ with a function exponent raised to the power of a linear polynomial.
                                            \[
                                            f(z) = e^{a+bz}\sin(\pi z)
                                            \]
                                            If $b\neq 0$, then the function will no longer be odd (e.g. consider $a=0, z=\pm i$). Hence 
                                            \[
                                            f(z) = e^{a}\sin(\pi z)
                                            \]
                                            so $f(z)$ can by any nonzero constant multiple of $\sin(\pi z)$.
                                            \end{solution}
                                            \end{document}
