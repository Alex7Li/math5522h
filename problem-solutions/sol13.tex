\documentclass{homework}
\course{Math 5522H}
\author{Jim Fowler}
\usepackage{amsmath}
\DeclareMathOperator{\Mat}{Mat}
\DeclareMathOperator{\End}{End}
\DeclareMathOperator{\Hom}{Hom}
\DeclareMathOperator{\id}{id}
\DeclareMathOperator{\image}{im}
\DeclareMathOperator{\rank}{rank}
\DeclareMathOperator{\nullity}{nullity}
\DeclareMathOperator{\trace}{tr}
\DeclareMathOperator{\Spec}{Spec}
\DeclareMathOperator{\Sym}{Sym}
\DeclareMathOperator{\pf}{pf}
\DeclareMathOperator{\Ortho}{O}
\DeclareMathOperator{\diam}{diam}
\DeclareMathOperator{\Real}{Re}
\DeclareMathOperator{\Imag}{Im}
\DeclareMathOperator{\Arg}{Arg}
\DeclareMathOperator{\Log}{Log}

\newcommand{\C}{\mathbb{C}}
\newcommand{\R}{\mathbb{R}}
\newcommand{\Z}{\mathbb{Z}}
\newcommand{\N}{\mathbb{N}}


\DeclareMathOperator{\sla}{\mathfrak{sl}}
\newcommand{\norm}[1]{\left\lVert#1\right\rVert}
\newcommand{\transpose}{\intercal}

\newcommand{\conj}[1]{\overline{#1}}
\newcommand{\abs}[1]{\left|#1\right|}

%%% My commands, for solutions %%%

\usepackage{amssymb}
\usepackage{xifthen}
\usepackage{listings}
\usepackage{tikz} % Guide http://bit.ly/gNfVn9
\usetikzlibrary{decorations.markings}
\DeclareMathOperator{\Res}{Res}

% To write df/(dx), use \pfrac{f}{x}
\newcommand{\pfrac}[2]{\frac{\partial #1}{\partial #2}}
% Partial derivative. To take d^2f/(dxdy), use \ppfrac[y]{f}{x}
% To take d^2f/(dx^2), use ppfrac{f}{x}
\newcommand{\ppfrac}[3][]{\frac{\partial^2 #2}{\ifthenelse{\isempty{#1}}{\partial #3^2}{\partial #3\partial #1}}}
\newcommand{\oo}[0]{\infty}

 \newenvironment{solution}
   {\renewcommand\qedsymbol{$\blacksquare$}\begin{proof}[Solution]}
     {\end{proof}}
       
       % Code listing environment  
       \lstnewenvironment{code}{\lstset{basicstyle=\ttfamily, mathescape=true, breaklines=true}}{}

       % When you want to see how many pages your HW is
       \usepackage{lastpage}
       \usepackage{fancyhdr}
       \pagestyle{fancy} 
       \cfoot{\thepage\ of \pageref{LastPage}}




\usepackage[symbol]{footmisc}
\renewcommand{\thefootnote}{\fnsymbol{footnote}}

\begin{document}
\maketitle

% https://www.maths.tcd.ie/pub/HistMath/People/Riemann/Grund/Grund.pdf
% page 37
\begin{inspiration}
  Zwei gegebene einfach zusammenh\"angende ebene Fl\"achen k\"onnen
    stets so auf einander bezogen werden, dass jedem Punkte der einen
      Ein mit ihm stetig fortr\"uckender Punkt der anderen entspricht und
        ihre entsprechenden kleinsten Theile \"ahnlich sind; und zwar kann
          zu Einem innern Punkte und zu Einem Begrenzungspunkte der
            entsprechende beliebig gegeben werden; dadurch aber ist f\"ur all
              Punkte die Beziehung bestimmt.\footnote{``Two given simply-connected
                  plane Surfaces can always be related to each other, so that each
                      Point in the One with a continuously advancing Point corresponds
                          to the other and their corresponding smallest Parts are similar
                              [meaning the map is conformal?]; and namely, the correspondence
                                  between Inner Points and Limit Points can be given arbitrarily;
                                      but because of that the Relationship for all Points is
                                          determined.''}  \byline{Bernhard Riemann, in his 1851 PhD thesis
                                              \textit{Grundlagen f\"ur eine allgemeine Theorie der Functionen
                                                    einer ver\"anderlichen complexen Gr\"osse.}}
                                                    \end{inspiration}

                                                    \section{Terminology}

                                                    \begin{problem}
                                                      What is a \textbf{conformal map}?  Recall
                                                        \ref{holomorphic-is-conformal} and note that various people define
                                                          ``conformal map'' differently.
                                                          \end{problem}
                                                          \begin{solution}
                                                          A conformal map can be defined as a map $f$ so that for any two curves $\gamma_1, \gamma_2:(-1, 1) \to \C$ intersecting at time $0$, the angle between $\gamma_1$ and $\gamma_2$ at time zero is equal in magnitude to the angle between the curves $f\circ\gamma_1, f\circ \gamma_2$ formed by their images under $f$ at time $0$.
                                                              
                                                              Another definition is to say that a map $f$ is conformal if it is bijective and both holomorphic or antiholomorphic.
                                                              \end{solution}
                                                              \begin{problem}
                                                                What is a \textbf{conformal equivalence} between two open subsets of $\C$?
                                                                \end{problem}
                                                                \begin{solution}
                                                                Open sets $U, V\subseteq \C$ are said to be conformally equivalent if there exists a conformal function $f:U\to V$.
                                                                \end{solution}
                                                                \begin{problem}
                                                                  How does the cross-ratio extend to $\hat{\C} = \C \cup \{\infty\}$?
                                                                    Recall \ref{cross-ratio}.
                                                                    \end{problem}
                                                                    \begin{solution}
                                                                    The cross ratio 
                                                                    \[
                                                                    (A,B;C,D) = \frac{(C-A)(D-B)}{(C-B)(D-A)}
                                                                    \]
                                                                    is defined for any tuple of 4 distinct points in $\C$. If one of the points is infinity, then we replace it with the number $N$ and take the limit as $N$ goes to infinity along the real axis,effectively cancelling out the terms involving it. For example,
                                                                    \[
                                                                    (A,B;C,\infty) = \lim_{N\to\infty} \frac{(C-A)(N-B)}{(C-B)(N-A)} = \frac{C-A}{C-B} 
                                                                    \]
                                                                    \end{solution}

                                                                    \section{Numericals}

                                                                    Feel free to express your
                                                                    biholomorphisms as the composition of other maps.

                                                                    \begin{problem}
                                                                      Consider the sector \(S_\theta := \{ z \in \C : 0 < |z| < 1 \mbox{ and } 0 < \Arg z < \theta \}\). Describe a biholomorphic map between $S_\theta$ and $S_\pi$.
                                                                      \end{problem}
                                                                      \begin{solution}
                                                                      We will define $f:S_\theta \to S_\pi$ by 
                                                                      $$f(z) = z^\frac{\pi}{\theta}  = e^{\frac{\pi}{\theta}\log z}.$$

                                                                      This function is holomorphic since it's the composition of $e^z$, multiplication, and log of $z$, which is defined on a simply connected domain not including 0. The function is injective since $0<\theta<2\pi$, and it has inverse 
                                                                      $$f^{-1}(z) = z^\frac{\theta}{\pi}$$
                                                                      which is also injective when the domain is $S_\pi$.
                                                                      \end{solution}
                                                                      \begin{problem}\label{halfdisktoquadrant}
                                                                        Consider the half-disk
                                                                          \[
                                                                              D := \{ x + iy \in \C : x, y \in \R \mbox{ and } x^2+y^2 < 1 \mbox{ and } x > 0 \}
                                                                                \]
                                                                                  and the first quadrant
                                                                                    \[
                                                                                        Q := \{ x + iy \in \C : x, y \in \R \mbox{ and } x > 0 \mbox{ and } y > 0 \}.
                                                                                          \]
                                                                                            Describe a biholomorphic map between $D$ and $Q$.
                                                                                            \end{problem}
                                                                                            \begin{solution}
                                                                                            Let $\conj{D}$ be the lower half disk. First we construct a map $g: D\to\conj{D}$ by $g(z)=ze^{\pi\theta}$. This is simply a rotation by 180 degrees, so the image of the upper half disk will in fact be the lower half disk. This holomorphic function is an involution, so it is biholomorphic.

                                                                                            Let 
                                                                                            \[
                                                                                            f : Q \to \conj{D} \text{ be defined as } f(z) = \frac{z - i}{z + i}.
                                                                                            \]
                                                                                            Since this function is a mobius transfomation $\frac{az+b}{cz+d}$ with $ad-bc\neq 0$, so it is biholomorphic.
                                                                                            First we check that the image of $f$ is contained in $D$. The norm of points in the image is less than one since for any point with positive imaginary part, $\abs{z-i}\leq \abs{z + i}$. We also note that the argument of $\frac{z-i}{z+i}$ is equal to the angle between the point $z-i$, the origin, and $z+i$, which can be seen geometrically be be between 0 and $2\pi$.

                                                                                            Next, note that $f$ has inverse given by 
                                                                                            \[
                                                                                            f^{-1}(z) = \frac{-iz-i}{z-1}
                                                                                            \]
                                                                                            and we can verify this
                                                                                            \begin{align*}
                                                                                            f(f^{-1}(z)) &= f(\frac{-iz-i}{z-1}) \\
                                                                                            &=
                                                                                            \frac{\frac{-iz-i}{z-1} - i}{\frac{-iz-i}{z-1} + i}\\
                                                                                            &= 
                                                                                            \frac{(-iz-i) - (iz - i)}{(-iz -i) + (iz - i)}\\
                                                                                            &= \frac{-2iz}{-2i} = z
                                                                                            \end{align*}
                                                                                            Finally, we need to check that the image of $f^{-1}(D)$ is contained in $Q$. 
                                                                                            \[
                                                                                            f^{-1}(z) = -i\frac{z+1}{z-1}
                                                                                            \]
                                                                                            we can see that $\frac{z+1}{z-1}$ has negative real part since the numerator is positive and the denominator is negative. The imaginary part is positive since $z+1$ is at a greater clockwise angle than $z-1$ for points below the $x$ axis. Then multiplying by $-i$, we get a number with positive real and imaginary parts. Hence we have confirmed that $f$ is a biholomorphism with the correct domain and codomain, and $e^{\pi \theta}f(z)$ is a map between $D$ and $Q$.

                                                                                            \end{solution}
                                                                                            \begin{problem}
                                                                                              Again consider the half-disk $D$. Describe a biholomorphic map between $D$ and $B_1(0)$.
                                                                                              \end{problem}
                                                                                              \begin{solution}

                                                                                              We will first use the function $f_1(z): D \to B_1\setminus [0, 1) = z^2.$  This is holomorphic with inverse $z^{-\frac{1}{2}}$.

                                                                                              Next we use the function 
                                                                                              \[
                                                                                              f_2(z): B_1\setminus [0, 1) \to \{ x + iy \in \C : x, y \in \R \mbox{ and } 0 < y < 2\pi \} = \log(z),
                                                                                              \]
                                                                                              where $\log(z)$ is choosen to have a branch cut on the positive real axis. Then $\log(re^{i\theta}) = \log(r)  + i\theta$. Taking $\theta \in (0, 2\pi)$, we see that the image of $f_2(z)$ is the codomain. The inverse of log is $\exp$, and we can verify that the image of the codomain under $\exp$ is $B_1\setminus [0, 1)$.

                                                                                              Now we can scale and shift this region using the obvious biholomorphisms to get the region defined in the next problem. So, it suffices to solve the next problem, then compose all of these maps together.
                                                                                              \end{solution}
                                                                                              \begin{problem}
                                                                                                Consider the infinite horizontal strip
                                                                                                  \[
                                                                                                      S := \{ x + iy \in \C : x, y \in \R \mbox{ and } |y| < 1 \}.
                                                                                                        \]
                                                                                                          Describe a biholomorphic map between $S$ and $B_1(0)$.  
                                                                                                          \end{problem}
                                                                                                          \begin{solution}
                                                                                                          First we shift the region with the biholomorphic function 
                                                                                                          \[
                                                                                                          f_1(z):S\to \{ x + iy \in \C : x, y \in \R \mbox{ and } 0 < y < \pi \} =  (z+1)\pi/2.
                                                                                                          \]

                                                                                                          Then we can apply the function 
                                                                                                          \[
                                                                                                          f_2(z):  \{ x + iy \in \C : x, y \in \R \mbox{ and } 0 < y < \pi \} \to \{x+iy \in C: y > 0\} = e^{z}
                                                                                                          \]
                                                                                                          to the image.

                                                                                                          The inverse of this function is $\log z$, and we can check that the image is the codomain because the point $x+iy$ maps to the point with magnitude $e^x$ and angle $0<y<\pi$, which is the whole top half plane. So it is biholomorphic.

                                                                                                          Then we want to map the half plane to the unit circle by again applying the mobius transformation considered in problem 5,
                                                                                                          \[
                                                                                                          \frac{z-i}{z+i}
                                                                                                          \]
                                                                                                          Problem \ref{halfdisktoquadrant} shows that this function is biholomorphic with inverse $-i\frac{z+1}{z-1}$.
                                                                                                          We can see that the image of the half plane is contained in the unit circle since when the imaginary part of $z$ is positive, $\abs{z-i} < \abs{z+i}$. The image of the unit disk under the inverse has positive imaginey part: $\frac{z+1}{z-1}$ is a positive number divided by a negative one and hence has negative real part. After multiplying by $-i$, the imaginary part is negative. Hence the tranformation sends the half plane to the unit circle, and by composing the above transformations, we get a biholomorphism from $S$ to $B_1(0)$.
                                                                                                          \end{solution}
                                                                                                          \section{Exploration}

                                                                                                          \begin{problem}
                                                                                                            There are 24 permutations of four distinct objects, and sometimes
                                                                                                              permuting the four inputs to the cross-ratio results in a different
                                                                                                                output.  Using the same four inputs, how many different outputs of
                                                                                                                  the cross-ratio are possible?  How are these outputs related to each
                                                                                                                    other?  There is a group structure here, giving rise to the
                                                                                                                      \textbf{anharmonic group}.
                                                                                                                      \end{problem}
                                                                                                                      \begin{solution}
                                                                                                                      \[
                                                                                                                      (A,B;C,D) = \frac{(C-A)(D-B)}{(C-B)(D-A)}
                                                                                                                      \]
                                                                                                                      Swapping both ($A$ and $C$) and ($B$ and $D$) at the same time does not change either the numerator or denominator, halving the maximum number of distict possibilities.

                                                                                                                      Swapping $A$ and $B$ swaps the numerator and denominator, as does swapping $C$ and $D$, again halving the maximum number of distinct possibilities.

                                                                                                                      So there are a maximum of 6 ways. Note that the operations described above do not change the set containing (the sets of elements of the first two slots) and (the set of elements of the last two slots). We can check that these three values can be different by letting $A=1, B=2, C=4, D=8$:

                                                                                                                      \begin{align*}
                                                                                                                      (A,B;C,D) &= \frac{(C-A)(D-B)}{(C-B)(D-A)} &= \frac{3\cdot 6}{2\cdot 7} &= \frac{9}{7}\\
                                                                                                                      (A,C;B,D) &= \frac{(B-A)(D-C)}{(B-C)(D-A)} &= \frac{1\cdot 4}{-2\cdot 7} &= \frac{-2}{7}\\
                                                                                                                      (A,D;B,C) &= \frac{(C-A)(B-D)}{(C-D)(B-A)} &= \frac{3 \cdot -6}{-4 \cdot 1} &= \frac{9}{2}
                                                                                                                      \end{align*}
                                                                                                                      As noted, swapping the first two elements takes the inverse of each of these quantities, so we have a total of 6 distinct values for the cross product.

                                                                                                                      \end{solution}

                                                                                                                      \begin{problem}\label{entire-injective-is-affine}A nonconstant entire function
                                                                                                                        $f : \C \to \C$ is \textit{affine} if there are constants
                                                                                                                          $a \in \C \setminus \{0\}$ and $b \in \C$ so that $f(z) = az + b$.
                                                                                                                            
                                                                                                                              Show that injective entire functions are affine.  \textit{Hint:}
                                                                                                                                apply Casorati-Weierstrass to $z \mapsto f(1/z)$.
                                                                                                                                \end{problem}
                                                                                                                                \begin{solution}
                                                                                                                                We note that an injective entire function has exactly one zero. Now by Weirstrauss's factorization theorem, any entire function with zeros at the point $0$ can be written as $g(z) = ze^{h(z)}$ with $h(z)$ entire. If $h(z)$ is constant, then this function is affline. Otherwise, $h_2(z) := h(1/z)$ has a pole or essential singularity at zero. If the singularity is essential, then $e^{h_2(z)}$ also has an essential singularity at 0, since by Casorati-Weirstrauss, $h_2(z)$ is dense in $\C$ at the point 0, so $e^{h_2(z)}$ is too, so $\lim{z\to 0} z^ne^{h_2(z)}$ will never converge for any $n$.

                                                                                                                                Otherwise, $h_2$ has a pole of degree $m$ at 0, then suppose that $e^{h_2(z)}$ is a pole of degree $n$. Then
                                                                                                                                \[
                                                                                                                                \pfrac{e^{h_2(z)}}{z} = h_2'(z)e^{h_2(z)}
                                                                                                                                \]
                                                                                                                                is a pole of degree $n+1$ by considering taylor series. But the expression on the right is a pole of degree $(m + 1) + n$, so $m=0$, a contradiction. So the singularity is essential in this case as well.

                                                                                                                                Now we can consider the function 
                                                                                                                                \[
                                                                                                                                g(\frac{1}{z}) = \frac{e^{h(\frac{1}{z})}}{z}
                                                                                                                                \]
                                                                                                                                This function has an essential singularity at 0, so it cannot be injective: If a function has an essential singularity, then Casorati-Weierstrass's theorem states that it assumes every possible complex value but one infinitely often in a neighborhood of 0. Thus $g(z) = ze^c$ for some constant $c$, so any injective entire function with a zero at 0 is affline. If the zero is at the point $w$ rather than 0, the function can be written as $g(z-w)$, which is also affline.
                                                                                                                                \end{solution}
                                                                                                                                \begin{problem}
                                                                                                                                  Using \ref{entire-injective-is-affine} and
                                                                                                                                    \ref{automorphisms-of-disk} and some dimension counting, conclude
                                                                                                                                      there is no biholomorphism between the unit ball $B_1(0)$ and the
                                                                                                                                        plane $\C$.  (Of course, this is \textit{much easier} to see via
                                                                                                                                          Liouville!)
                                                                                                                                          \end{problem}
                                                                                                                                          \begin{solution}
                                                                                                                                          We recall that all automorphisms of the unit disks can be indexed by a complex parameter $\abs{w} < 1$ and a real parameter $\theta$ and written as
                                                                                                                                          \[
                                                                                                                                          f(z) = e^{i\theta}\frac{z-w}{1 - z\conj{w}}
                                                                                                                                          \]
                                                                                                                                          Letting $w=a+bi$, we can see that this group of automorphisms forms 3 dimensional manifold over $\R$ with the following metric
                                                                                                                                          \[
                                                                                                                                          d\left(e^{i\theta_0}\frac{z-(a_0+b_0i)}{1 - z(a_0-b_0i)}), e^{i\theta_1}\frac{z-(a_1+b_1i)}{1 - z(a_1-b_1i)}\right) = 
                                                                                                                                          \abs{a_0-a_1} + \abs{b_0-b_1} + \min_{k\in \Z} \abs{\theta_0-\theta_1-2\pi k}
                                                                                                                                          \]
                                                                                                                                          For any $a_0 + ib_0 \in B_1(0), \theta_0\in \R$, we can choose an $\epsilon$ so that the set of automorphisms with distance less than $\epsilon$ is a ball in $\R^3$, so this set of automorphisms is a 3 dimensional real manifold.

                                                                                                                                          Next, note that by the precceding problem, all automorphisms of the complex plane are affline functions $az+b, a\neq 0$. By a similar choice of metric, we can see that the set of automorphisms forms a 4 dimensional real manifold.

                                                                                                                                          Now suppose that there is a biholomorphism between the unit ball and $\C$. Then we have a homeomorphism from a 3 dimensional real manifold into a 4 dimesional real manifold, but this is impossible.
                                                                                                                                          \end{solution}
                                                                                                                                          \begin{problem}
                                                                                                                                            Define \(B_{r,R}(0) := \{ z \in \C : r < |z| < R \}\). 
                                                                                                                                              Suppose $0 < r < R$.  Is there a biholomorphism between the
                                                                                                                                                punctured disk $B_{0,R}(0)$ and the annulus $B_{r,R}(0)$?
                                                                                                                                                \end{problem}
                                                                                                                                                \begin{solution}
                                                                                                                                                There cannot be such a biholomorphism. If $f:B_{0, R} \to B{r, R}$ were such a holomorphism, then it has an isolated singularity at 0. 
                                                                                                                                                As the image of $f$ is bounded by the ball of radius $R$, the singularity is removable and we can extend $f$ to a function $\hat{f}$ from $B_{R} \to B{r, R}$. Now, $f(B_{r/2})$ is an open set by the open mapping theorem, so $\hat{f}(0)$ is not on the boundary of the annulus. 

                                                                                                                                                Thus $\hat{f}$ is not injective, and $\hat{f}(0) = \hat{f}(z)$ for some $z\in B_{0, R}$. Considering the image of a small ball around $0$ and a disjoint small ball around $z$, they are both open sets containing $f(0)$ by the open mapping theorem, and thus have a intersection of some open set. This implies that there are more points on which $\hat{f}$ is not injective, so $f$ is not injective.

                                                                                                                                                \end{solution}
                                                                                                                                                \begin{problem}\label{annulus-scale}
                                                                                                                                                  Is there a biholomorphism between the annulus $B_{r,R}(0)$ and the
                                                                                                                                                    annulus $B_{\lambda r, \lambda R}(0)$?
                                                                                                                                                    \end{problem}
                                                                                                                                                    \begin{solution}
                                                                                                                                                    Yes, define 
                                                                                                                                                    \[
                                                                                                                                                    f(z):B_{r, R}(0) \to B_{\lambda r , \lambda R}(0) = \lambda z.
                                                                                                                                                    \]

                                                                                                                                                    We can check that the domain and range match up since this is just a scaling and rotation. Provided that $\lambda$ is not zero, this function admits an inverse (multiplying by $\frac{1}{\lambda}$, so it is a biholomorphism iif $\lambda\neq 0.$

                                                                                                                                                    \end{solution}
                                                                                                                                                    \begin{problem}\label{annulus-flip}
                                                                                                                                                      Is there a biholomorphism between the annulus $B_{r,R}(0)$ and the
                                                                                                                                                        annulus $B_{1/R,1/r}(0)$?
                                                                                                                                                        \end{problem}
                                                                                                                                                        \begin{solution}
                                                                                                                                                        Yes, we can use the function
                                                                                                                                                        \[
                                                                                                                                                        f(z): B_{r, R}(0) \to B_{\frac{1}{R}, \frac{1}{r}}(0) = \frac{1}{z}
                                                                                                                                                        \]
                                                                                                                                                        The function $\frac{1}{z}$ is holomorphic and an involution, and it's easy to see that the function is bijective onto it's codomain.
                                                                                                                                                        \end{solution}

                                                                                                                                                        \begin{problem}\label{annulus-to-punctured-disk}Suppose $r > 1$ and $R > 1$ and that there is a biholomorphism
                                                                                                                                                          $f : B_{1,r}(0) \to B_{1,R}(0)$ sending the inside boundary to the
                                                                                                                                                            inside boundary, i.e., for a sequence
                                                                                                                                                              $z_1, z_2, \ldots \in B_{1,r}(0)$ with $\lim_n |z_n| = 1$, we have
                                                                                                                                                                $\lim_n |f(z_n)| = 1$.
                                                                                                                                                                  
                                                                                                                                                                    Repeatdly reflect across the boundary to produce a biholomorphism
                                                                                                                                                                      $F : \C \setminus \{0\} \to \C \setminus \{0\}$ extending $f$.

                                                                                                                                                                        \textit{A technical point:} the Schwarz reflection principle, as we
                                                                                                                                                                          have stated it, requires $f$ to extend continuously to the boundary.
                                                                                                                                                                            This issue can be overcome here.
                                                                                                                                                                            \end{problem}
                                                                                                                                                                            \begin{solution}
                                                                                                                                                                            First, let's try to extend $f$ continuously to it's boundary. Define
                                                                                                                                                                            \[
                                                                                                                                                                            \overline{f(z)}:\overline{B_{1,r}(0)}\to \overline{B_{1,R}(0)} = \begin{cases}
                                                                                                                                                                            f(z) & 1 < z < r \\ 
                                                                                                                                                                            \lim_{w\to z} f(w) & |z| = 1 \lor |z| = r
                                                                                                                                                                            \end{cases}
                                                                                                                                                                            \]
                                                                                                                                                                            Since $f$ is holomorphic hence continuous, and bounded above by $R$, the limit always exists, so $\overline{f(z)}$ is always defined. In addition to being holomorphic on the interior, it is continuous on the boundary, since for any point $z_0$ on the boundary, for any $\delta$, since $f$ is continuous, we can an $\epsilon$ so that each point within a ball of radius $\epsilon$ is within $\delta$ of $z_0.$

                                                                                                                                                                            We will define a function $g:\overline{B_{r,r^2}}(0) \to \overline{B_{R, R^2}}(0)$ that extends $f$.
                                                                                                                                                                            To do this, we take an input $z\in \overline{B_{r, R}(0)}$, invert it's conjugate, and then scale/shift it to be in the domain of $f$, $\overline{B_{1,r}(0)}$ by the function 
                                                                                                                                                                            \[
                                                                                                                                                                            z \to \frac{r^2}{\conj{z}}
                                                                                                                                                                            \]
                                                                                                                                                                            Then we can apply $f$ to get something with the image $\overline{B_{1,R}(0)}$. We invert it's conjugate to make it holomorphic. Then we can then scale by $R^2$ to get something with the image $\overline{B_{1, R^2}(0)}$. In total,
                                                                                                                                                                            \[
                                                                                                                                                                            g(z) = R^2/\conj{f(\frac{r^2}{\conj{z}})}
                                                                                                                                                                            \]
                                                                                                                                                                            $g(z)$ is the composite of conformal maps, so it is conformal. Since exactly two of the maps are complex conjugation, the sign of the angle is switched twice, and thus $g(z)$ is holomorphic.

                                                                                                                                                                            Now we can try to extend $\hat{f}$ to $h(z):\overline{B_{1, r^2}} \to\overline{B_{1, R^2}}$ by

                                                                                                                                                                            \[
                                                                                                                                                                            h(z) = \begin{cases}
                                                                                                                                                                            \hat{f}(z) & |z| \leq r\\
                                                                                                                                                                            g(z) & |z| \geq r\\
                                                                                                                                                                            \end{cases}
                                                                                                                                                                            \]
                                                                                                                                                                            We first note that the two functions agree on the boundary, since if $|z|=r$ then $r^2/\conj{z} = z$ so $g(z) = R^2/\conj{f(z)}$, and since $f$ maps the outer boundary of $f$ to the outer boundary, $f(z)\conj{f(z)} = R^2$. Then we can apply the Schwarz reflection principal to see that $h(z)$ is holomorphic. Technically this principal is for the half plane, but the proof also works in the annulus case: considering any closed curve in the region, we can split it into closed curves above and below the circle with radius $r$ and see that the integral of $f$ along these curves is zero. This function is bijective since $\hat{f}$ is bijective onto $\overline{B_{1, R}(0)}$ and $g(z)$ is bijective onto $\overline{B_{R, R^2}(0)}$.


                                                                                                                                                                            We can repeat this argument with $h(z)$ to find a function $\overline{B_{1, r^4}} \to\overline{B_{1, R^4}}$, and repeat it on this function and so on until we have a function $f_1$ defined on the entire complex plane other than the unit ball. Finally, we can define a function
                                                                                                                                                                            \[
                                                                                                                                                                            f_2(z): \C\setminus \{0\}\to \C\setminus \{0\} =
                                                                                                                                                                            \begin{cases}
                                                                                                                                                                            f_1(z)& |z| \geq 1\\
                                                                                                                                                                            1/\conj{f_1(\frac{1}{\conj{z}})}&  |z| \leq 1\\
                                                                                                                                                                            \end{cases}
                                                                                                                                                                            \]
                                                                                                                                                                            The same proof as above shows that $f_2(z)$ is everywhere holomorphic and continuous on $|z|=1$, and again applying the Schwarz reflection principal, we can conclude that $f_2(z)$ is holomorphic. Since $f_1(z)$ is bijective onto all the numbers with magnitude at least 1, $1/\conj{f_1(\frac{1}{\conj{z}})}$ is bijective onto all the numbers with magnitude at most 1, with equality at the boundary. Thus $f_2$ is bijective, giving us the desired automorphic biholomorphism on $\C\setminus \{0\}$ extending $f$.
                                                                                                                                                                            \end{solution}
                                                                                                                                                                            \begin{problem}
                                                                                                                                                                              The function $F$ constructed in \ref{annulus-to-punctured-disk}
                                                                                                                                                                                extends to an entire function, which by
                                                                                                                                                                                  \ref{entire-injective-is-affine} is affine.  Deduce
                                                                                                                                                                                    \textbf{Schottky's theorem for annuli} describing exactly when
                                                                                                                                                                                      annuli $B_{r,R}(0)$ and $B_{r',R'}(0)$ and are conformally
                                                                                                                                                                                        equivalent.
                                                                                                                                                                                        \end{problem}
                                                                                                                                                                                        \begin{solution}
                                                                                                                                                                                        We see that the function in \ref{annulus-to-punctured-disk} extends to an entire function with a zero at the point 0. As an entire injective function with a root at 0, it must be of the form $cz$ with $0\neq c \in\C$. Since the aforementioned function maps the unit circle to the unit circle, $c$ is a complex number with magnitude 1.

                                                                                                                                                                                        Suppose that we have $f_1:B_{r,R}(0) \to B_{r',R'}(0)$ a conformal mapping. Then we can apply the scaling conformal map considered in \ref{annulus-scale} to the output and input to get a conformal map $f_2:B_{1,\frac{R}{r}}(0) \to B_{1,\frac{R'}{r'}}(0)$. By the previous paragraph, $f_2$ is a rotation, so we must have that $\frac{R}{r} = \frac{R'}{r'}$, and this is clearly a sufficient condition since the identity map is a conformal map between the two.
                                                                                                                                                                                        \end{solution}
                                                                                                                                                                                        \section{Prove or Disprove and Salvage if Possible}

                                                                                                                                                                                        \begin{problem}
                                                                                                                                                                                          A map $f : \C \to \C$ is conformal if and only if it is holomorphic.
                                                                                                                                                                                          \end{problem}
                                                                                                                                                                                          \begin{solution}
                                                                                                                                                                                          False. For one thing, $f$ may be $z\to \conj{z}$, though some definitions of conformal will not allow for this function. Other than that, $z^2$ is holomorphic on $\C$, but this is not conformal since the angle between vectors starting at the point $0$ get ruined in the transformation (or perhaps because $z^2$ is not injective, depending on your definition).

                                                                                                                                                                                          We salvage as follows:

                                                                                                                                                                                          Any (holomorphic or antiholomorphic) and bijective map $f : \C \to \C$ is conformal.

                                                                                                                                                                                          If $f$ is (anti)holomorphic and bijective, we can define $f^{-1}(z)$, but it's not obvious that this preserves angles.

                                                                                                                                                                                          Conside a curve passing through a point $w$ at this point, the Cauchy Riemann equations give

                                                                                                                                                                                          \[\frac{d}{dt} f(\gamma(t)) = 
                                                                                                                                                                                          \begin{pmatrix}
                                                                                                                                                                                          u_x & u_y\\
                                                                                                                                                                                          v_x & v_y\\
                                                                                                                                                                                          \end{pmatrix}
                                                                                                                                                                                          \begin{pmatrix}
                                                                                                                                                                                          \partial_1\gamma\\
                                                                                                                                                                                          \partial_2\gamma\\
                                                                                                                                                                                          \end{pmatrix}
                                                                                                                                                                                          = 
                                                                                                                                                                                          \begin{pmatrix}
                                                                                                                                                                                          u_x & u_y\\\
                                                                                                                                                                                          -u_y & u_x\\
                                                                                                                                                                                          \end{pmatrix}
                                                                                                                                                                                          \begin{pmatrix}
                                                                                                                                                                                          \partial_1\gamma\\
                                                                                                                                                                                          \partial_2\gamma\\
                                                                                                                                                                                          \end{pmatrix}\]

                                                                                                                                                                                          Provided that the derivative is not 0 at this point, we can choose some choice of $R, \theta$ based on the values of $u_x, u_y$ so that we get
                                                                                                                                                                                          \[\frac{d}{dt} f(\gamma(t)) = 
                                                                                                                                                                                          R
                                                                                                                                                                                          \begin{pmatrix}
                                                                                                                                                                                          \cos{\theta} & \sin{\theta}\\
                                                                                                                                                                                          -\sin{\theta} & \cos{\theta}\\
                                                                                                                                                                                          \end{pmatrix}
                                                                                                                                                                                          \begin{pmatrix}
                                                                                                                                                                                          \partial_1\gamma\\
                                                                                                                                                                                          \partial_2\gamma\\
                                                                                                                                                                                          \end{pmatrix}
                                                                                                                                                                                          \]
                                                                                                                                                                                          In the antiholomorphic case, the angle $-\theta$ will work instead, since $u_y$ and $-u_y$ are switched. 

                                                                                                                                                                                          Thus, applying $f$ to $\gamma$ at the point $w$ will give us a derivative vector pointing in the direction $\arg(\gamma) \pm \theta,$ and so for any two curves passing through $\gamma$, the difference of the angles two curves after the transformation is the same as before.

                                                                                                                                                                                          Thus $f$ is conformal provided that $f'$ is never 0. Bijectiveness enforces this, since if $f'(w)= 0$, we can write out it's local power series to find that $c_n(z-w)^n + g(z)$ with $g(z)\in O(z-w)^{n+1}$ and $n\geq 2$. Then we can find a disk of radius $r$ around $w$ on which $g(z) < c_n(z-w)^n$, and by Rouche's theorem,
                                                                                                                                                                                          $f(z)-f(w)-\frac{r^n}{2}$ attains as many zeros on this ball as $(z-w)^n-\frac{r^2}{2}$ does, a total of $n\geq 2$ zeros, contradicting bijectivity.
                                                                                                                                                                                          \end{solution}
                                                                                                                                                                                          \begin{problem}
                                                                                                                                                                                            For triples $z_1, z_2, z_3 \in \C$ and $w_1,w_2,w_3 \in \C$, there
                                                                                                                                                                                              is a conformal map $f : \C \to \C$ with $f(z_1) = w_1$ and
                                                                                                                                                                                                $f(z_2) = w_2$ and $f(z_3) = w_3$. % many issues!
                                                                                                                                                                                                \end{problem}
                                                                                                                                                                                                \begin{solution}
                                                                                                                                                                                                False, suppose that the $z_i$ lie on a line, but the $w_i$ do not. By problem \ref{entire-injective-is-affine}, all entire conformal maps are affline, so the image of points lying in a line are on a line, meaning there is no such $f$ in this case. 

                                                                                                                                                                                                If we replace each instance of $\C$ with the Riemann sphere $\hat{\C}$, the statement becomes true. To see this, first note that mobius transformations are conformal: $\frac{az+b}{cz+d}$ with $ac-bd\neq 0$ is a composition of conformal maps from $\hat{C} \to \hat{C}$ if we consider $\frac{1}{z}$ to be conformal in the sense that it maps 0 to infinity.

                                                                                                                                                                                                To satisfy the condition that $f(z) = w$ with a mobius transformations, we need to find $a,b,c,d$ so that $az + b = w(cz + d)$. Thus we can solve it iff we can solve the system of linear equations
                                                                                                                                                                                                \begin{align*}
                                                                                                                                                                                                    \begin{pmatrix}
                                                                                                                                                                                                        z_1&1&-w_1z_1&-w_1\\
                                                                                                                                                                                                            z_2&1&-w_2z_2&-w_2\\
                                                                                                                                                                                                                z_3&1&-w_3z_3&-w_3\\
                                                                                                                                                                                                                    \end{pmatrix}
                                                                                                                                                                                                                        \begin{pmatrix}
                                                                                                                                                                                                                            a\\b\\c\\d\\
                                                                                                                                                                                                                                \end{pmatrix}
                                                                                                                                                                                                                                    =
                                                                                                                                                                                                                                        \begin{pmatrix}
                                                                                                                                                                                                                                            0\\0\\0\\0\\
                                                                                                                                                                                                                                                \end{pmatrix}
                                                                                                                                                                                                                                                \end{align*}
                                                                                                                                                                                                                                                We can put the LHS in rref form and find a solution to this no matter what.
                                                                                                                                                                                                                                                \end{solution}
                                                                                                                                                                                                                                                \begin{problem}
                                                                                                                                                                                                                                                  If there is a biholomorphism $f : U \to V$ and $U$ is
                                                                                                                                                                                                                                                    simply-connected, then $V$ is simply-connected.
                                                                                                                                                                                                                                                    \end{problem}
                                                                                                                                                                                                                                                    \begin{solution}
                                                                                                                                                                                                                                                    This is true since any biholomorphism is a homeomorphism.
                                                                                                                                                                                                                                                    \end{solution}
                                                                                                                                                                                                                                                    \end{document}

                                                                                                                                                                                                                                                    
