\documentclass{homework}
\course{Math 5522H}
\author{Alex Li}
\usepackage{amsmath}
\DeclareMathOperator{\Mat}{Mat}
\DeclareMathOperator{\End}{End}
\DeclareMathOperator{\Hom}{Hom}
\DeclareMathOperator{\id}{id}
\DeclareMathOperator{\image}{im}
\DeclareMathOperator{\rank}{rank}
\DeclareMathOperator{\nullity}{nullity}
\DeclareMathOperator{\trace}{tr}
\DeclareMathOperator{\Spec}{Spec}
\DeclareMathOperator{\Sym}{Sym}
\DeclareMathOperator{\pf}{pf}
\DeclareMathOperator{\Ortho}{O}
\DeclareMathOperator{\diam}{diam}
\DeclareMathOperator{\Real}{Re}
\DeclareMathOperator{\Imag}{Im}
\DeclareMathOperator{\Arg}{Arg}
\DeclareMathOperator{\Log}{Log}

\newcommand{\C}{\mathbb{C}}
\newcommand{\R}{\mathbb{R}}
\newcommand{\Z}{\mathbb{Z}}
\newcommand{\N}{\mathbb{N}}


\DeclareMathOperator{\sla}{\mathfrak{sl}}
\newcommand{\norm}[1]{\left\lVert#1\right\rVert}
\newcommand{\transpose}{\intercal}

\newcommand{\conj}[1]{\overline{#1}}
\newcommand{\abs}[1]{\left|#1\right|}

%%% My commands, for solutions %%%

\usepackage{amssymb}
\usepackage{xifthen}
\usepackage{listings}
\usepackage{tikz} % Guide http://bit.ly/gNfVn9
\usetikzlibrary{decorations.markings}
\DeclareMathOperator{\Res}{Res}

% To write df/(dx), use \pfrac{f}{x}
\newcommand{\pfrac}[2]{\frac{\partial #1}{\partial #2}}
% Partial derivative. To take d^2f/(dxdy), use \ppfrac[y]{f}{x}
% To take d^2f/(dx^2), use ppfrac{f}{x}
\newcommand{\ppfrac}[3][]{\frac{\partial^2 #2}{\ifthenelse{\isempty{#1}}{\partial #3^2}{\partial #3\partial #1}}}
\newcommand{\oo}[0]{\infty}

 \newenvironment{solution}
   {\renewcommand\qedsymbol{$\blacksquare$}\begin{proof}[Solution]}
     {\end{proof}}
       
       % Code listing environment  
       \lstnewenvironment{code}{\lstset{basicstyle=\ttfamily, mathescape=true, breaklines=true}}{}

       % When you want to see how many pages your HW is
       \usepackage{lastpage}
       \usepackage{fancyhdr}
       \pagestyle{fancy} 
       \cfoot{\thepage\ of \pageref{LastPage}}




\begin{document}
\maketitle

\begin{inspiration}
  \textit{Logic sometimes makes monsters.} For half a century we have
    seen a mass of bizarre functions which appear to be forced to
      resemble as little as possible honest functions which serve some
        purpose. More of continuity, or less of continuity, more
          derivatives, and so forth.
            \byline{Henri Poincar\'e} % can someone find the actual reference for this?!
            \end{inspiration}

            \section{Terminology}

            \begin{problem}
              What is meant by uniform convergence on compact sets?
              \end{problem}
              \begin{solution}
              A sequence of functions $f_n:A \to B$ is said to uniformly converge on compact sets if the limiting function $\lim_{n\to\infty} f_n = f$ exists and for any compact subset $S\subset A$, the sequence restricted to $S$ converges uniformly to $f$ restricted to $f$. 
              That is, for any $\delta>0$ we can find an $\epsilon>0$, $n$ so that 
              \[\forall s_1, s_2\in S,\, d(s_1, s_2) \leq \epsilon \implies d(f_n(s_1), f(s_2)) \leq \delta.\]
              \end{solution}
              \begin{problem}
                What is a biholomorphism $f : U \to V$?
                \end{problem}
                \begin{solution}
                $f$ is said to be a biholomorphism if there exists a holomorphic function $g$ such that $f\circ g = g\circ f = Id$. 
                \end{solution}
                \begin{problem}
                  Define \textbf{star-convex}.
                  \end{problem}
                  \begin{solution}
                  A set $U\in \C$ is said to be start-convex if $\exists z \in U$ such that for any point $w \in U$, for any $t\in [0, 1]$, $zt + w(1-t) \in U$.
                  \end{solution}

                  \section{Numericals}

                  \begin{problem}\label{max-modulus-numerical}
                    Consider $f(z) = 4z - (2z - i)^2 - 2i - 4$.  Find $z$ in 
                      \[
                          \{ x + iy \in \C : 0 \leq x \leq 1 \mbox{ and } 0 \leq y \leq 1 \}
                            \]
                              maximizing $\abs{f(z)}$.
                              \end{problem}
                              \begin{solution}
                              First we can try to expand $f$
                              \begin{align*}
                              f(x+ iy) = (-4 x^2 + 4 x + 4 y^2 - 4y - 3) + (-8 x y +  4 x + 4y - 2)i
                              \end{align*}

                              By the maximum modulus principal %\ref{maximum-modulus-principle}
                              , the maximum value of $\abs{f(z)}$ will occur on the boundary of the region. So we need to find the max over 4 lines:
                              \begin{align*}
                              f_1(x) &= (-4 x^2 + 4 x - 3) + (4 x - 2)i\quad \color{purple} y=0\\ 
                              f_2(y) &=  (4 y^2 - 4y - 3) + (4y - 2)i \quad \color{purple} x=0\\
                              f_3(x) &=  (-4 x^2 + 4 x - 3) + (-4 x + 2)i \quad \color{purple} y=1 \\
                              f_4(y) &=  (4 y^2 - 4y - 3) + (-4y + 2)i \quad \color{purple} x=1
                              \end{align*}
                              These are really similar, after squaring the $i$ term is equal and thus we want to choose signs so as to maximize
                              \begin{align*}
                              \abs{f(t)} &= (\pm 4(t^2 - t) -3)^2 + (4t - 2)^2
                              \end{align*}
                              Since $4(t^2 - t)$ is always negative and less than 3 in magnitude for $0\leq t \leq 1$, the maximum will come on when we choose the plus sign.
                              \begin{align*}
                              \abs{f(t)} &= (4(t^2 - t) -3)^2 + (4t - 2)^2\\
                              &= 16(t^2 - t)^2 - 24(t^2 - t)  + 9+ 16(t^2 - t) + 4\\
                              &= 16(t^2 - t)^2 - 8(t^2 - t) + 13
                              \end{align*}
                              When $t\in [0, 1]$, $t^2 - t\in [-\frac{1}{4}, 0]$, and letting $u = f(t^2-t)$, the maximum comes when $u$ is at the endpoints or a local extrema. The values at the endpoints are $13$ and $16$. There are no local extrma of $u$ on the interval- at such a point, the derivative would be in the interval
                              \begin{align*}
                              \pfrac{f}{u} &= \pfrac{}{u}16u^2 - 8u + 13\\
                              0 &= 8u + 16\\
                              u &= 2
                              \end{align*}
                              Thus the maximum value is $16$, and it occurs when $t^2-t=\frac{-1}{4}$, so when $t = .5$, which corresponds to either the point $.5i$ or $1+.5i$
                              \end{solution}
                              \section{Exploration}

                              \begin{problem}\label{uniform_limit_is_holomorphic}
                                Sometimes things are \textit{much} better in the complex case than
                                  the real case.  Find a sequence of smooth functions
                                    $f_n : [-1,1] \to \R$ converging uniformly to $f(x) = \abs{x}$ which
                                      is not differentiable at zero.  Can the uniform limit of holomorphic
                                        functions fail to be holomorphic?
                                        \end{problem}
                                        \begin{solution}
                                        We can use fourier series to find such a sum of functions. Since the function is odd, the nth coefficient is 
                                        \[\int_{-1}^1 f(x)\cos(n\pi x)x dx = 2((-1)^n-1)\cos(\pi nx).\] Since $f(x)$ is continuous and $f'(x)$ exists at all by finitely many points and is square integrable, the Dirichlet conditions for Fourier series imply that the following sequence converges uniformly to $f(x)$ on $[-1, 1]$:
                                        \begin{align*}
                                            f_n = \frac{1}{2} + \sum_{n=1}^n \frac{2((-1)^n-1)\cos(\pi nx)}{(n\pi)^2}
                                            \end{align*}

                                            For holomorphic functions, any sequence will again converge to a holomorphic function. Let $g_n$ be a sequence of holomorphic functions with limiting function $g$. Then for each $g_n$, the integral around any curve $\gamma$ is 0. Taking the limit, 
                                            \begin{align*}
                                                \int_\gamma g &= \int_\gamma \lim_{n\to\infty} g_n dz\\
                                                    &= \lim_{n\to\infty}\int_\gamma  g_n dz \quad \color{purple} \text{Uniform Convergence}\\
                                                        &= \lim_{n\to\infty} 0 = 0
                                                        \end{align*}
                                                        Thus the integral around any curve is 0, so there is a primitive for $g$, and that primitive has $g$ as a derivative so it is holomorphic, so we can take a second derivative and thus $g$ is holomorphic as well.
                                                        \end{solution}
                                                        \begin{problem}\label{hadamard-three-lines}Prove \textbf{Hadamard's
                                                            three-lines theorem} which is about the strip
                                                              \[
                                                                  S_{a,b} := \{ x+iy \in \C : a \leq x \leq b\}
                                                                    \]
                                                                      bounded by two horizontal lines, i.e., the line with real part $a$
                                                                        and the line with real part $b$.  Suppose $f : S_{a,b} \to \C$ is a
                                                                          bounded continuous function which is holomorphic on the interior of
                                                                            $S_{a,b}$.  Define $M : [a,b] \to \R$ by
                                                                              \[
                                                                                  M(x) := \sup_{y \in \R} \abs{f(x+iy)},
                                                                                    \]
                                                                                      that is, $M(x)$ is the supremum of $\abs{f}$ on the (third!) line with real part $x$.
                                                                                        Show that, if $t \in [0,1]$, then \[
                                                                                            M\left( ta + (1-t)b \right) \leq M(a)^t M(b)^{1-t}.
                                                                                              \]
                                                                                              \end{problem}
                                                                                              \begin{solution}
                                                                                              WLOG let $a=0, b=1$. We will split this problem into cases:
                                                                                              \begin{enumerate}
                                                                                                  \item $M(0) = M(1) = 1$\\
                                                                                                          By boundedness, $\exists\, M\, s.t. \forall\, z,\, f(z) < M.$
                                                                                                                  
                                                                                                                          For any $t>0$, we define an auxilary function on any point in $S_{1,0}$ by $g(z) = \frac{f(z)}{1 + tz}$. $g(z)$ is holomorphic and less than $f(z)$.
                                                                                                                                  
                                                                                                                                          We know that 
                                                                                                                                                  \[g(z) \leq \frac{M}{1 + t|y|}\]
                                                                                                                                                          
                                                                                                                                                                  Choose $d$ to be big enough that $\frac{M}{1+td} < 1$, then consider the rectangular contour with $\{x+iy\in C: x=0\lor x= 1\lor |y| = d\}.$ The maximum value of $g(z)$ on this contour is achieved on the boundary, so it is at most 1.    By increasing $d$, the maximum value of $g(z)$ is 1. 
                                                                                                                                                                          
                                                                                                                                                                                  Then letting $t$ go to zero, $g(z)\to f(z)$ so $f(z)$ has maximum value 1.
                                                                                                                                                                                      \item $M(0) > 0, M(1) > 0$
                                                                                                                                                                                              Let 
                                                                                                                                                                                                      \[g(x+iy) = f(x+iy)M(0)^{x-1}M(1)^{-x}.\] 
                                                                                                                                                                                                              This is holomorphic and the conditions from the first case apply: when $x=0$,
                                                                                                                                                                                                                      \[g(x+iy)\leq \frac{f(x+iy)}{M(0)} \leq 1,\] and when $x=1$, \[g(x+iy)\leq \frac{f(x+iy)}{M(1)} \leq 1.\]
                                                                                                                                                                                                                              Thus we conclude that $g(x+iy) \leq 1$. Therefore
                                                                                                                                                                                                                                      \[f(x+iy) = g(x+iy)M(0)^{1-x}M(1)^{x}\leq M(0)^{1-x}M(1)^{x} \]
                                                                                                                                                                                                                                              Thus $M(1-t) \leq M(0)^tM(1)^{1-t}$, as desired.
                                                                                                                                                                                                                                                  \item $M(0)=0$ or $M(1) = 0$
                                                                                                                                                                                                                                                      
                                                                                                                                                                                                                                                              We can use the last argument, but replacing the relevant terms with $\epsilon>0$ to show that $M(1-t) \leq (M(0)+\epsilon)^t (M(1)+\epsilon)^{1-t}$, and thus $M(1-t) = 0$.
                                                                                                                                                                                                                                                              \end{enumerate}

                                                                                                                                                                                                                                                              \end{solution}
                                                                                                                                                                                                                                                              \begin{problem}\label{schwarz-lemma}Suppose $f : B_1(0) \to B_1(0)$ is holomorphic and $f(0) = 0$.  Show
                                                                                                                                                                                                                                                                that $\abs{f(z)} \leq \abs{z}$ for all $z \in B_1(0)$.  How large
                                                                                                                                                                                                                                                                  could $\abs{f'(0)}$ be?  This is the \textbf{Schwarz lemma}.
                                                                                                                                                                                                                                                                  \end{problem}
                                                                                                                                                                                                                                                                  \begin{solution}
                                                                                                                                                                                                                                                                  Define 
                                                                                                                                                                                                                                                                  \[
                                                                                                                                                                                                                                                                  g(z) = \begin{cases}\frac{f(z)}{z} & z \neq 0 \\ f'(0) & z= 0\end{cases}
                                                                                                                                                                                                                                                                  \]
                                                                                                                                                                                                                                                                  By considering the power series expansion for $f$, we see that $g$ is everywhere holomorphic:
                                                                                                                                                                                                                                                                  \[
                                                                                                                                                                                                                                                                  f(z) =  \sum_{n=1}^\infty a_nz^n \implies f'(z) = \sum_{n=1}^\infty na_nz^{n-1} \implies f'(0) = a_1 
                                                                                                                                                                                                                                                                  \]
                                                                                                                                                                                                                                                                  and 
                                                                                                                                                                                                                                                                  \[
                                                                                                                                                                                                                                                                  \lim_{z\to 0} f(z)/z = \lim_{z\to 0} \sum_{n=1}^\infty a_nz^{n-1} = a_1 
                                                                                                                                                                                                                                                                  \]

                                                                                                                                                                                                                                                                  Now consider values of $g$ on the boundary of the disk of radius $r$ centered at the origin. By the maximum principal, the maximum value of $g$ occurs on the boundary. Consdering it's codomain, $\abs{f(z)} \leq 1$, so for all $z\neq 0$,
                                                                                                                                                                                                                                                                  \[
                                                                                                                                                                                                                                                                  \abs{g(z)} = \abs{\frac{f(z)}{z}} \leq \abs{\frac{1}{r}}.
                                                                                                                                                                                                                                                                  \]
                                                                                                                                                                                                                                                                  Letting $r\to 1$, we see that $\abs{g(z)}\leq 1$ and therefore \[\abs{f(z)} \leq \abs{z}.\]
                                                                                                                                                                                                                                                                  This implies that
                                                                                                                                                                                                                                                                  \[\abs{\frac{f(z)}{z}} \leq 1,\]
                                                                                                                                                                                                                                                                  so for all $z\neq 0$, $\abs{g(z)}\leq 1$, and by continuity, $\abs{g(0)}\leq 1$, so $\abs{f'(0)}\leq 1$.
                                                                                                                                                                                                                                                                  \end{solution}
                                                                                                                                                                                                                                                                  \begin{problem}\label{schwarz-lemma-2}Suppose again that $f : B_1(0) \to B_1(0)$ is holomorphic, $f(0) = 0$, and
                                                                                                                                                                                                                                                                    suppose further that $\abs{f(z_0)} = \abs{z_0}$ for some nonzero
                                                                                                                                                                                                                                                                      $z_0 \in \C$.  Show that $f$ must be a rotation, i.e., there is some
                                                                                                                                                                                                                                                                        $\lambda \in \C$ with $\abs{\lambda} = 1$ and $f(z) = \lambda z$.
                                                                                                                                                                                                                                                                          (This is also the Schwarz lemma.)
                                                                                                                                                                                                                                                                          \end{problem}
                                                                                                                                                                                                                                                                          \begin{solution}
                                                                                                                                                                                                                                                                          Consider again the holomorphic function defined in the last problem
                                                                                                                                                                                                                                                                          \[
                                                                                                                                                                                                                                                                          g(z) = \begin{cases}\frac{f(z)}{z} & z \neq 0 \\ f'(0) & z= 0\end{cases}
                                                                                                                                                                                                                                                                          \]
                                                                                                                                                                                                                                                                          $\abs{g(z)}\leq 1$, and at the point $z_0$, it is equal to 1 by assumption. By the maximum principal, it is equal to $1$ everywhere on the unit ball.
                                                                                                                                                                                                                                                                          \end{solution}
                                                                                                                                                                                                                                                                          \begin{problem}\label{schwarz-lemma-3}Suppose that
                                                                                                                                                                                                                                                                            $f : B_1(0) \to B_1(0)$ is holomorphic and $\abs{f'(0)} = 1$.  What
                                                                                                                                                                                                                                                                              can you deduce about $f$?
                                                                                                                                                                                                                                                                              \end{problem}
                                                                                                                                                                                                                                                                              \begin{solution}
                                                                                                                                                                                                                                                                              Since $f$ is holomorphic, it is equal to it's power series in the ball it is defined on. Suppose that the power series has more than 1 nonzero coefficient. Let $n$ be the index of the second nonzero coefficient of the power series, so
                                                                                                                                                                                                                                                                              \[\frac{f(z)}{a_1} = z + a_nz^n + \sum_{m>n}^\infty a_mz^m\]
                                                                                                                                                                                                                                                                              Now if $a_n = r_2e^{i\theta_2}$, choose $z_0 = \epsilon e^{\frac{i\theta_2}{1-n}}$. Then 
                                                                                                                                                                                                                                                                              \begin{align*}
                                                                                                                                                                                                                                                                              \frac{f(z_0)}{a_1} &= \epsilon e^{\frac{i\theta_2}{1-n}} + \epsilon^n r_2e^{i\theta_2}e^{\frac{ni\theta_2}{1-n}} + \sum_{m>n}^\infty a_mz^m\\
                                                                                                                                                                                                                                                                              \frac{f(z_0)}{a_1} &= (\epsilon+r_2\epsilon^n)e^{\frac{i\theta_2}{1-n}} + \sum_{m>n}^\infty a_mz^m\\
                                                                                                                                                                                                                                                                              \frac{f(z_0)}{a_1}  &= \abs{(\epsilon+r_2\epsilon^n)e^{\frac{i\theta_2}{1-n}} + \sum_{m>n}^\infty a_mz^m}\\
                                                                                                                                                                                                                                                                              \frac{f(z_0)}{a_1}  &\geq \epsilon+r_2\epsilon^n - \sum_{m>n}^\infty \abs{a_m}\epsilon^{m}
                                                                                                                                                                                                                                                                              \end{align*}
                                                                                                                                                                                                                                                                              And letting $\epsilon\to 0$, all the terms in the sum die out and since $r_2>0$ we get that $f(\epsilon e^{\frac{i\theta_2}{1-n}}) > \epsilon$, a contradiction. Therefore the power series for $f$ only has one nonzero coefficient, and $f = a_1z$ for some $a_1$ with $\abs{a_1} = 1$.
                                                                                                                                                                                                                                                                              \end{solution}
                                                                                                                                                                                                                                                                              \begin{problem}\label{schwarz-reflection-principle-2}Consider the
                                                                                                                                                                                                                                                                                closed upper half-plane
                                                                                                                                                                                                                                                                                  \[
                                                                                                                                                                                                                                                                                      H := \{ x + iy \in \C : x \in \R \mbox{ and } y \geq 0 \}.
                                                                                                                                                                                                                                                                                        \]
                                                                                                                                                                                                                                                                                          Suppose $f : H \to \C$ is continuous, holomorphic on the interior of
                                                                                                                                                                                                                                                                                            $H$, and sends reals to reals, i.e., if $x \in \R$ then
                                                                                                                                                                                                                                                                                              $f(x) \in \R$.  Use a trick like \ref{schwarz-reflection-principle}
                                                                                                                                                                                                                                                                                                to describe an entire function agreeing with $f$ on its domain.
                                                                                                                                                                                                                                                                                                \end{problem}
                                                                                                                                                                                                                                                                                                \begin{solution}
                                                                                                                                                                                                                                                                                                Define $\hat{f}(z) = 
                                                                                                                                                                                                                                                                                                \begin{cases}
                                                                                                                                                                                                                                                                                                    \conj{f(\conj{z})} & x < 0\\
                                                                                                                                                                                                                                                                                                        f(z)&x\geq 0
                                                                                                                                                                                                                                                                                                        \end{cases}$
                                                                                                                                                                                                                                                                                                        This function is continuous everywhere because $f$ agrees with $\conj{f(\conj{z})}$ on the real line, and holomorphic everywhere except the real line. We will show it is also holomorphic on that line.

                                                                                                                                                                                                                                                                                                        Let $\gamma$ be an arbitrary rectangle, it suffices to show that
                                                                                                                                                                                                                                                                                                        \[\int_{\gamma} f(z)dz = 0.\]
                                                                                                                                                                                                                                                                                                        If $\gamma$ does not cross the real line, this is true by the holomorphisity of $\hat{f}$ everywhere not real. If $\gamma$ has complex part greater (or similarly less) than or equal to 0 everywhere, then we know that
                                                                                                                                                                                                                                                                                                        \[\int_{\gamma} f(z+\epsilon i)dz = 0\]
                                                                                                                                                                                                                                                                                                        and letting $\epsilon\to 0$, continuity implies that
                                                                                                                                                                                                                                                                                                        \[\int_{\gamma} f(z)dz = 0.\]
                                                                                                                                                                                                                                                                                                        Finally, if $\gamma$ crosses the real line, we can split it into the sum of the rectangle below and above the real line, and so the integral of $\gamma$ is 0. By Morera's theorem, $f$ is holomorphic.
                                                                                                                                                                                                                                                                                                        \end{solution}
                                                                                                                                                                                                                                                                                                        \begin{problem}
                                                                                                                                                                                                                                                                                                          Consider the closed disk
                                                                                                                                                                                                                                                                                                            \[
                                                                                                                                                                                                                                                                                                                D := \{ z \in \C : \abs{z} \leq 1 \},
                                                                                                                                                                                                                                                                                                                  \]
                                                                                                                                                                                                                                                                                                                    and suppose $f : D \to D$ is continuous, holomorphic on the interior
                                                                                                                                                                                                                                                                                                                      of $D$, sends boundary to boundary, i.e., if $z \in \partial D$ then
                                                                                                                                                                                                                                                                                                                        $f(z) \in \partial D$, and misses an interior point, i.e., there is
                                                                                                                                                                                                                                                                                                                          some $w$ in the interior of $D$ which is not in the image of $f$.

                                                                                                                                                                                                                                                                                                                            Combine \ref{cayley-transform} and
                                                                                                                                                                                                                                                                                                                              \ref{schwarz-reflection-principle-2} to produce an entire function.
                                                                                                                                                                                                                                                                                                                                What can you deduce about $f$?
                                                                                                                                                                                                                                                                                                                                \end{problem}
                                                                                                                                                                                                                                                                                                                                \begin{solution}
                                                                                                                                                                                                                                                                                                                                Consider the mobius transformation $M(z) = \frac{iz+1}{z + i}$, we saw in an earlier problemset that $M(z)$ sends the open disk to the half plane and is holomorphic. %\ref{cayley-transform}
                                                                                                                                                                                                                                                                                                                                Therefore $M\circ f$ is a function satisfying the hypotheses of \ref{schwarz-reflection-principle-2}, so we can create an entire function by the extension $\hat{M\circ f}(z) = 
                                                                                                                                                                                                                                                                                                                                \begin{cases}
                                                                                                                                                                                                                                                                                                                                    \conj{M\circ f(\conj{z})} & x < 0\\
                                                                                                                                                                                                                                                                                                                                        M\circ f(z)&x\geq 0
                                                                                                                                                                                                                                                                                                                                        \end{cases}$. 
                                                                                                                                                                                                                                                                                                                                        Now, neither $M(w)$ or $\conj{M(w)}$ is in the image of $D$, and these are not the same point since $w$ is in the interior of $D$. By Picard's little theorem, $f$ is constant, and since $M\circ f =\conj{M\circ f(\conj{z}}$ on the boundary, $f(z)$ is equal to some constant value with modulus $1$.
                                                                                                                                                                                                                                                                                                                                        \end{solution}
                                                                                                                                                                                                                                                                                                                                        \section{Prove or Disprove and Salvage if Possible}

                                                                                                                                                                                                                                                                                                                                        \begin{problem}\label{cauchy-for-starlike}Suppose $U \subset \C$ is a
                                                                                                                                                                                                                                                                                                                                          star-convex open set, and $f : U \to \C$ is holomorphic, and
                                                                                                                                                                                                                                                                                                                                            $\gamma : [0,1] \to U$ is a smooth closed curve.  Then
                                                                                                                                                                                                                                                                                                                                              \[
                                                                                                                                                                                                                                                                                                                                                  \int_\gamma f(z) \, dz = 0
                                                                                                                                                                                                                                                                                                                                                    \]
                                                                                                                                                                                                                                                                                                                                                    \end{problem}
                                                                                                                                                                                                                                                                                                                                                    \begin{solution}
                                                                                                                                                                                                                                                                                                                                                    This is true by Cauchy's theorem. Since $U$ is star convex, we can find a point in U so that the line segement from that point to any other point in $U$ is contained in $U$. WLOG let this point be 0.
                                                                                                                                                                                                                                                                                                                                                    For any closed curve $\gamma$, we have a homotopy $f:[0, 1]\times [0, 1] \to \C$ given by 
                                                                                                                                                                                                                                                                                                                                                    \[
                                                                                                                                                                                                                                                                                                                                                    f(a, b) = \gamma(a)/b
                                                                                                                                                                                                                                                                                                                                                    \]
                                                                                                                                                                                                                                                                                                                                                    By inspection, this function is continuous, $f(t, 1)=\gamma(t)$, and $f(\gamma(t),0)= 0$, so the curves $\gamma(t)$ and $0$ are homotopic. Now we can apply Cauchy's theorem and say that the given integral is 0 for any holomorphic function.
                                                                                                                                                                                                                                                                                                                                                    \end{solution}
                                                                                                                                                                                                                                                                                                                                                    \begin{problem}\label{moreras-theorem}If a function $f : B_r(0) \to \C$
                                                                                                                                                                                                                                                                                                                                                      satisfies % missing continuous!
                                                                                                                                                                                                                                                                                                                                                        \[
                                                                                                                                                                                                                                                                                                                                                            \int_\gamma f(z) \, dz = 0
                                                                                                                                                                                                                                                                                                                                                              \]
                                                                                                                                                                                                                                                                                                                                                                for all piecewise smooth closed curves $\gamma$ in the disk
                                                                                                                                                                                                                                                                                                                                                                  $B_r(0)$, then $f$ is holomorphic.
                                                                                                                                                                                                                                                                                                                                                                  \end{problem}
                                                                                                                                                                                                                                                                                                                                                                  \begin{solution}
                                                                                                                                                                                                                                                                                                                                                                  No, consider $f = \begin{cases}10 & z=0\\ 0 & z\neq 0\end{cases}$. The integral of any closed curve is 0 since all but measure 0 points are 0, and yet $f$ is not holomorphic it is not even continuous.

                                                                                                                                                                                                                                                                                                                                                                  Suppose that $f$ is continuous as well. Then we can find a primitive $F$ for $f$, and by the fundamental theorem of calculus, $\pfrac{F}{z} = f$. Thus $F$ is holomorphic and we can take a second derivative, which will be the derivative of $f$, so $f$ is also holomorphic.
                                                                                                                                                                                                                                                                                                                                                                  \end{solution}

                                                                                                                                                                                                                                                                                                                                                                  \begin{problem}\label{uniform-convergence-holomorphic}If the sequence of
                                                                                                                                                                                                                                                                                                                                                                    holomorphic functions $f_n : U \to \C$ converge pointwise to
                                                                                                                                                                                                                                                                                                                                                                      $f : U \to \C$, then $f$ is holomorphic.
                                                                                                                                                                                                                                                                                                                                                                      \end{problem}
                                                                                                                                                                                                                                                                                                                                                                      \begin{solution}
                                                                                                                                                                                                                                                                                                                                                                      No, consider $f_n = x^n$ on the closed ball $U = B_{1}(0)$. This converges pointwise, but the limiting function 
                                                                                                                                                                                                                                                                                                                                                                      \[f(re^{i\theta}) = \begin{cases} 0 & r<1\\ 1 & r=1 \end{cases}\]
                                                                                                                                                                                                                                                                                                                                                                      is not even continuous. 

                                                                                                                                                                                                                                                                                                                                                                      To salvage this, suppose that $f_n$ converges absolutely. Then by \ref{uniform_limit_is_holomorphic}, $f$ is holomorphic.
                                                                                                                                                                                                                                                                                                                                                                      \end{solution}

                                                                                                                                                                                                                                                                                                                                                                      \begin{problem}\label{automorphisms-of-disk}Suppose $f, g : B_r(0) \to B_r(0)$ are holomorphic and
                                                                                                                                                                                                                                                                                                                                                                        $f \circ g = g \circ f$ are the identity on $B_r(0)$.  Then there is
                                                                                                                                                                                                                                                                                                                                                                          $w \in B_r(0)$ and $\theta \in [0,2\pi)$ with
                                                                                                                                                                                                                                                                                                                                                                            \[
                                                                                                                                                                                                                                                                                                                                                                                f(z) = e^{i\theta} \cdot \frac{w - z}{1 - \conj{w}z}.
                                                                                                                                                                                                                                                                                                                                                                                  \]
                                                                                                                                                                                                                                                                                                                                                                                    This should remind you of \ref{blaschke-factors}.  This result
                                                                                                                                                                                                                                                                                                                                                                                      highlights the rigidity of biholomorphic functions as compared to,
                                                                                                                                                                                                                                                                                                                                                                                        say, homeomorphisms.
                                                                                                                                                                                                                                                                                                                                                                                        \end{problem}
                                                                                                                                                                                                                                                                                                                                                                                        \begin{solution}
                                                                                                                                                                                                                                                                                                                                                                                        Since $f$ is invertible (with inverse $g$, $\exists! \,w \,s.t. \,f(w) = 0$.

                                                                                                                                                                                                                                                                                                                                                                                        Notice that the function $M(z) = \frac{w-z}{1-\conj{w}z}$ is a mobius transformation, and hence it is holomorphic and invertible. Thus $f\circ M^{-1}$ has inverse $M \circ g$ and since $M(w)=0$, both this function and it's inverse have a zero at $z=0$. Then by the schwarz lemma \ref{schwarz-lemma}, $\abs{(f\circ M^{-1})(z)} \leq \abs{z}$ and $\abs{(M\circ g)(z)} \leq \abs{z}$.
                                                                                                                                                                                                                                                                                                                                                                                        As these two functions are inverses, we must in fact have equality: $\abs{(f\circ M^{-1})(z)} = \abs{z}$. Now using another form of the schwarz lemma \ref{schwarz-lemma-2}, we can conclude that $(f\circ M^{-1})(z) = e^{i\theta}z$, and therefore that
                                                                                                                                                                                                                                                                                                                                                                                        \[f = e^{i\theta}\frac{w-z}{1-\conj{w}z}\]
                                                                                                                                                                                                                                                                                                                                                                                        % :)
                                                                                                                                                                                                                                                                                                                                                                                        \end{solution}
                                                                                                                                                                                                                                                                                                                                                                                        \end{document}
