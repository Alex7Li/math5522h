\documentclass{homework}
\course{Math 5522H}
\author{Alex Li}
\usepackage{amsmath}
\DeclareMathOperator{\Mat}{Mat}
\DeclareMathOperator{\End}{End}
\DeclareMathOperator{\Hom}{Hom}
\DeclareMathOperator{\id}{id}
\DeclareMathOperator{\image}{im}
\DeclareMathOperator{\rank}{rank}
\DeclareMathOperator{\nullity}{nullity}
\DeclareMathOperator{\trace}{tr}
\DeclareMathOperator{\Spec}{Spec}
\DeclareMathOperator{\Sym}{Sym}
\DeclareMathOperator{\pf}{pf}
\DeclareMathOperator{\Ortho}{O}
\DeclareMathOperator{\diam}{diam}
\DeclareMathOperator{\Real}{Re}
\DeclareMathOperator{\Imag}{Im}
\DeclareMathOperator{\Arg}{Arg}
\DeclareMathOperator{\Log}{Log}

\newcommand{\C}{\mathbb{C}}
\newcommand{\R}{\mathbb{R}}
\newcommand{\Z}{\mathbb{Z}}
\newcommand{\N}{\mathbb{N}}


\DeclareMathOperator{\sla}{\mathfrak{sl}}
\newcommand{\norm}[1]{\left\lVert#1\right\rVert}
\newcommand{\transpose}{\intercal}

\newcommand{\conj}[1]{\overline{#1}}
\newcommand{\abs}[1]{\left|#1\right|}

%%% My commands, for solutions %%%

\usepackage{amssymb}
\usepackage{xifthen}
\usepackage{listings}
\usepackage{tikz} % Guide http://bit.ly/gNfVn9
\usetikzlibrary{decorations.markings}
\DeclareMathOperator{\Res}{Res}

% To write df/(dx), use \pfrac{f}{x}
\newcommand{\pfrac}[2]{\frac{\partial #1}{\partial #2}}
% Partial derivative. To take d^2f/(dxdy), use \ppfrac[y]{f}{x}
% To take d^2f/(dx^2), use ppfrac{f}{x}
\newcommand{\ppfrac}[3][]{\frac{\partial^2 #2}{\ifthenelse{\isempty{#1}}{\partial #3^2}{\partial #3\partial #1}}}
\newcommand{\oo}[0]{\infty}

 \newenvironment{solution}
   {\renewcommand\qedsymbol{$\blacksquare$}\begin{proof}[Solution]}
     {\end{proof}}
       
       % Code listing environment  
       \lstnewenvironment{code}{\lstset{basicstyle=\ttfamily, mathescape=true, breaklines=true}}{}

       % When you want to see how many pages your HW is
       \usepackage{lastpage}
       \usepackage{fancyhdr}
       \pagestyle{fancy} 
       \cfoot{\thepage\ of \pageref{LastPage}}




\begin{document}
\maketitle

\begin{inspiration}
I don't think \byline{Warren Buffett}
\end{inspiration}

\section{Terminology}

\begin{problem}
  What is a half-space?  Define convex hull.
\end{problem}
\begin{solution}
A (open) half-space is a set of points in $\C$ is the set of points on one side of a line, i.e. $H=\{z: z\cdot u > b\}$ for a unit vector $u\in\C$, $b\in\Z$
 
The convex hull of a set of points is the intersection of all half-spaces containing every point of the set.

\end{solution}
\begin{problem}
Define the Wirtinger partial derivatives $\displaystyle\frac{\partial}{\partial z}$ and $\displaystyle\frac{\partial}{\partial \conj{z}}$.
\end{problem}
\begin{solution}
For a function $f$ with domain $\C = \{x + iy|x, y\in\R\}$, we define the Wirtinger partial derivatives as follows:
\[\pfrac{}{z} = \frac{1}{2}(\pfrac{}{x} - i\pfrac{}{y})\]
\[\pfrac{}{\conj{z}} = \frac{1}{2}(\pfrac{}{x} + i\pfrac{}{y})\]
\end{solution}

\begin{problem}
  What are the Cauchy-Riemann equations?
\end{problem}
We can uniquely represent a complex function of $z = x + iy$ as a sum of two real functons of two variables, $u$ and $v$, to get the equation  $f(z) = u(x, y) + iv(x, y).$
The Cauchy-Riemann equations are the following
\begin{solution}
\[\pfrac{u}{x} = \pfrac{v}{y}\label{cauchy-riemann-x-real}\]
\[\pfrac{u}{y} = -\pfrac{v}{x}\label{cauchy-riemann-y-real}\]
\end{solution}
These two equations are neccessary for differentiability of f and, together with continuity of the partial derivatives, sufficient. 

\section{Numericals}

\begin{problem}\label{cauchy-riemann-polar}
What are the Cauchy-Riemann equations\ldots in polar coordinates?
\end{problem}
\begin{solution}
Suppose that $f(re^{i\theta}) = u(r, \theta) + iv(r, \theta)$ is a complex differentiable  function. Let's ignore the origin since $\theta$ isn't even defined, this formula can apply everywhere else. Then differenting along the first coordinate,
\begin{align*}
f'(r_0e^{i\theta_0}) &=  \lim_{r\to r_0} \frac{u(r, \theta_0) + iv(r, \theta_0) - u(r_0, \theta_0) - iv(r_0, \theta_0)}{re^{i\theta_0} - r_0e^{i\theta_0}}\\
&= \frac{1}{e^{i\theta_0}}\lim_{r\to r_0} \frac{u(r, \theta_0) - u(r_0, \theta_0)}{r - r_0} + \frac{i}{e^{i\theta_0}} \lim_{r\to r_0} \frac{v(r, \theta_0) - v(r_0, \theta_0)}{r - r_0}
\end{align*}
\begin{align}\label{first_rc_polar}
f'(r_0e^{i\theta_0}) &= \frac{u_r + iv_r}{e^{i\theta}}
\end{align}
Next, we differentiate along the second coordinate, $\theta$. 
\begin{align*}
f'(r_0e^{i\theta_0}) &=  \lim_{\theta\to \theta_0} \frac{u(r_0, \theta) + iv(r, \theta_0) - u(r_0, \theta_0) - iv(r_0, \theta_0)}{r_0(e^{i\theta} - e^{i\theta_0})}\\
&= \lim_{\theta\to \theta_0} \frac{u(r_0, \theta) - u(r_0, \theta_0)}{r_0e^{i}(e^\theta - e^{\theta_0})} + i \lim_{\theta\to \theta_0} \frac{v(r, \theta_0) - v(r_0, \theta_0)}{r_0e^{i}(e^\theta - e^{\theta_0})}
\end{align*}
We want to compare this to the real partial derivatives in the direction $\theta$:
\begin{gather*}
\pfrac{u(r_0, \theta_0)}{\theta} = \lim_{\theta\to \theta_0} \frac{u(r_0, \theta) - u(r_0, \theta_0)}{\theta - \theta_0}\\
\pfrac{v(r_0, \theta_0)}{\theta} = \lim_{\theta\to \theta_0} \frac{v(r_0, \theta) - v(r_0, \theta_0)}{\theta - \theta_0}
\end{gather*}
In the limit, the ratio between these two can be computed by noticing that $\lim_{\theta\to\theta_0}\frac{e^{i\theta} - e^{i\theta_0}}{i\theta - i\theta_0}$ is the derivative of $e^z$ at $i\theta_0$.
\[\lim_{\theta \to \theta_0} \frac{\theta - \theta_0}{r_0(e^{i\theta} - e^{i\theta_0})} = r_0\lim_{\theta \to \theta_0} \frac{i\theta - i\theta_0}{e^{i\theta} - e^{i\theta_0}}\frac{1}{i} = r_0\times e^{i\theta}\times\frac{1}{i}
\]
Thus we see that
\begin{align}\label{second_rc_polar}
f'(r_0e^{i\theta_0}) = \frac{v_{\theta_0} - iu_{\theta_0}}{e^{i\theta_0}r_0}
\end{align}
Setting real and imaginary parts equal in \ref{first_rc_polar} and \ref{second_rc_polar},
\begin{align*}
r_0u_r = v_\theta \\
r_0v_r = - u_\theta
\end{align*}
\end{solution}
\begin{problem}
  For a natural number $n$ define $f : \mathbb{C} \to \mathbb{C}$ by
  \[
    f(r \cos \theta + i \, r\sin \theta) = r^n \cos \left( n\theta \right) + i \, r^n \sin \left( n\theta \right).
  \]
  Use \ref{cauchy-riemann-polar} to verify that $f$ satisfies the
  Cauchy-Riemann equations.
\end{problem}
\begin{solution}
Let $u=r^n\cos(n\theta)$, $v = r^n\sin(n\theta)$, and we compute the 4 partials.
\begin{align*}
u_r &= nr^{n-1}\cos(n\theta)\\
v_r &= nr^{n-1}\sin(n\theta)\\
u_\theta &= -nr^{n}\sin(n\theta) = -rv_r\\
v_\theta &= nr^{n}\cos(n\theta) = ru_r\\
\end{align*}
\end{solution}


\begin{problem}
  On a certain domain, define $f(z) = \log \abs{z} + i \cdot \Arg z$
  and verify that $f$ satisfies the Cauchy-Riemann equations.
\end{problem}
\begin{solution}
Let $z=r\cos{\theta} + r\sin{\theta}$ with $\theta \in (-\pi, \pi)$, so $f(z) = \Log r + i \theta$. We validate the cauchy riemann equations on the domain $\C\setminus(-\infty, 0]$ by first noting that they are differentiable on this domain, then checking them in polar coordinates:
\begin{align*}
u_r &= \frac{1}{r} &  u_\theta &= 0\\
v_r &= 0 & v_\theta &= 1
\end{align*}
Thus $rv_r = u_\theta$ and $ru_r = v_\theta$
\end{solution}

\begin{problem}
Compute $\displaystyle\frac{\partial}{\partial z} \left( z \conj{z} \right)$.
\end{problem}
\begin{solution}
\begin{align*}
\pfrac{}{z} \left( z \conj{z} \right) &= \frac{1}{2}(\pfrac{}{x}(x^2+y^2) - i\pfrac{}{y}(x^2+y^2))\\
&= x - iy\\
\end{align*}
If we pretend $z$ and $\conj z$ are independent and differentiate, we get the same answer:
\begin{align*}
\pfrac{}{z} \left( z \conj{z} \right) &= \conj{z} = x - iy\\
\end{align*}
\end{solution}

\begin{problem}\label{harmonic-conjugate}Let $u(x,y) = e^y \sin x$.  Find $v(x,y)$ so that $u$ and $v$ satisfy the Cauchy-Riemann equations.  Such a function $v$ is a
  \textbf{harmonic conjugate} of $u$.
\end{problem}
\begin{solution}
First, let's compute the partial derivatives of $v$ with the Cauchy-Riemann equations.
\[u_x = -e^y\cos{x} = v_y\]
\[-u_y = -e^y\sin{x} = v_x\]
Integrating, 
\[v = \int v_y dy = -e^y\cos{x} + C(x)  = -e^y\cos{x} + C(y) = \int v_x dx\]
So $v = -e^y\cos{x} + C$ for any real number $C$.
\end{solution}
\section{Exploration}

\begin{problem}\label{cayley-transform}Relate the upper half-plane
  $H = \{ z \in \mathbb{C} : \Imag z > 0 \}$ and the open disk
  $D = \{ z \in \mathbb{C} : \abs{z} < 1 \}$ by finding $a,b,c,d \in \C$ so that the M\"obius transformation
  \[f(z) = \frac{az + b}{cz + d}\] yields a bijection $f : H \to D$.
\end{problem}
\begin{solution}
Maybe it will work if the transformation sends the real line to the unit circle, with say $f(0)\mapsto -i$, $f(1)\mapsto 1$, $f(\infty)\mapsto i$. Solving for $a,b,c,d$, we get the following mobius transformation:
$f(z) = \frac{iz+1}{z + i}.$

% https://en.wikipedia.org/wiki/M%C3%B6bius_transformation#Composition_of_simple_transformations
We can check that we can write this mobius tranformation $f(z) = f_4\cdot f_3 \cdot f_2 \cdot f_1(x)$ where $f_1 = z + i$ (mapping $H$ to the half plane above the line with i=1), $f_2 = 1/z$ (mapping $Image f_1$ to the open circle with radius $\frac{1}{2}$ between 0 and $-i$), $f_3 = 2z$ (mapping $Image f_2$ to the circle between 0 and $-2i$) and $f_4 = z + i$ (mapping $Image f_3$ to D).

It's easy to see that each of the $f_i$ are bijective, so $f$ must be bijective as well.
\end{solution}


\begin{problem}\label{harmonic-necessary}Given $u : \mathbb{R}^2 \to \mathbb{R}$, is it always possible to
  find a function $v : \mathbb{R}^2 \to \mathbb{R}$ so that $u$ and
  $v$ satisfy the Cauchy-Riemann equations?
\end{problem}
\begin{solution}
Obviously not if $u$ isn't differentiable. Let's assume that $u$ is twice differentiable with equal mixed partials. Then we can solve for $v$ like in \ref{harmonic-conjugate}:
\[v = \int u_x dy = -\int u_y dx.\]
Thus $v_{xy} = -u_{yy}$ and $v_{yx} = u_{xx}$, and we need $u$ to satisfy $u_{xx} = -u_{yy}$. If that's true, then integrating $u_xdy$ and $u_ydx$ should give the same answer for $v$, so $u$ will have a harmonic conjugate.
\end{solution}


\begin{problem}
  For $p \in \C[x]$, show that the convex hull of the zeroes of $p$
  contains the zeroes of $p'$.  This is the \textbf{Gauss-Lucas
    theorem}.
\end{problem}
\begin{solution}
By the fundamental theorem of algebra, we can split the polynomial $p$ into linear factors $p = \prod_i(z - r_i).$
Now notice that for functions $f_1(z), f_2(z), ... f_n(z)$, the product rule gives
\begin{align}\label{log-derivative-product-to-sum}
\frac{\frac{d}{dz}\prod_i f_i}{\prod_i f_i} = \frac{\sum_i f_i'\prod_{j\neq i}f_j}{\prod_i f_i} = \sum_i\frac{ f_i'}{f_i}
\end{align}
Using this fact, we can compute
\[
\frac{\frac{dp}{dz}}{p} = \frac{\frac{d}{dz}\prod_i (z - r_i)}{\frac{d}{dz}\prod_i (z - r_i)} = \sum_i \frac{\frac{d}{dz}(z-r_i)}{z - r_i} = \sum_i \frac{1}{z - r_i}
\]

Let's consider an arbitrary half space $H=\{z: z\cdot u > b\}$ for a unit vector $u\in\C$, $b\in\Z$ which contains none of the zeros of $p$. Then for points $z$ in this half space, $(z-r_i)\cdot u = z\cdot u - r_i \cdot u$.
Here $z\cdot u > b$ since $z\in H$, and $r_i \cdot u <= b$ since $r_i\not\in H$, so $(z-r_i)\cdot u > 0$ and thus $\frac{1}{z-r_i}\cdot u > 0$.

By linearity, \[0 < \sum_i \frac{1}{z - r_i} \cdot u  = \frac{\frac{dp}{dz}}{p}\cdot u\] so $\frac{dp}{dz} \neq 0$ on this half space (as $p$ is nonzero). Since the half space was choosen aribtrarily from the half spaces containing none of the zeros of $p$, there are no zeros of $p'$ outside of the convex hull of zeros of $p$.
\end{solution}
\begin{problem}\label{cauchy-riemann-again}
  Suppose the smooth function $f : \C \to \C$ satisfies the
  Cauchy-Riemann equations.  Does $f'$ also satisfy the Cauchy-Riemann
  equations?
\end{problem}
\begin{solution}
Let $f = u(x,y) + iv(x,y).$ 
Taking the derivative of $f$ in the $x$ direction, $f' = u_x + iv_x,$ and so to show $f'$ satisfies the Cauchy Riemann equations that we want to show that $u_{xx}=v_{xy}$ and $u_{xy} = -v_{xx}.$ To do this, notice that since $f$ satisfies the equations, $u_{x} = v_y$ and $u_y=-v_x$. Taking the $x$ derivative of these two equations and using the equality of mixed partials we show what we wanted: namely, that $u_{xx}=v_{xy}$ and $u_{xy} = -v_{xx}.$ So $f'$ satisfies the Cauchy Riemann equations.
\end{solution}
\begin{problem}
  Suppose $f : \R^2 \to \R^2$ satisfies the Cauchy-Riemann equations,
  and consider curves $\gamma_1, \gamma_2 : (-1,1) \to \R^2$ passing
  through the origin so that $\gamma_1(0) = \gamma_2(0) = (0,0)$.
  Relate the angle between the curves $\gamma_1$ and $\gamma_2$ to the
  angle between the curves $f \circ \gamma_1$ and $f \circ \gamma_2$.
  (This is related to \ref{tj-versus-jt}.)
\end{problem}
\begin{solution}
To get the angle between the curves, we need to compare the derivatives between the curves at the point (0,0).

\[\frac{d}{dt} f(\gamma_1(t)) = 
\begin{pmatrix}
u_x &u_y\\
v_x &v_y\\
\end{pmatrix}
\begin{pmatrix}
\partial_1\gamma_1\\
\partial_2\gamma_1\\
\end{pmatrix}
= 
\begin{pmatrix}
u_x & u_y\\\
-u_y & u_x\\
\end{pmatrix}
\begin{pmatrix}
\partial_1\gamma_1\\
\partial_2\gamma_1\\
\end{pmatrix}\]
We can choose some choice of $R, \theta$ based on the values of $u_x, u_y$ so that we get
\[\frac{d}{dt} f(\gamma_1(t)) = 
R
\begin{pmatrix}
\cos{\theta} & \sin{\theta}\\
\sin{\theta} & \cos{\theta}\\
\end{pmatrix}
\begin{pmatrix}
\partial_1\gamma_1\\
\partial_2\gamma_1\\
\end{pmatrix}
\]
Thus, applying $f$ to $\gamma_1$ will give us a derivative vector pointing in the direction $\arg(\gamma_1) + \theta,$ and similarly applying $f$ to $\gamma_2$ will result in a derivative pointing in the direction $\arg(\gamma_2)+\theta$. Thus the argle between $\gamma_1$ and $\gamma_2$ at the point $(0,0)$ is equal to the angle after applying $f$.
\end{solution}
\begin{problem}
  Suppose $f : \C \to \C$ is smooth and holomorphic.  Show that the
  real part of $f$ is harmonic (cf.~\ref{harmonic-function}).
\end{problem}
\begin{solution}
We can solve similarly to \ref{cauchy-riemann-again}. Let $f=u+iv$, since $f$ is harmonic it satisfies the cauchy riemann equations and thus
\begin{align*}
u_x=\phantom{-}v_y &\implies u = \phantom{-}\int v_y dx &\implies u_{xx} = \phantom{-}v_{yx}\\
u_y=-v_x &\implies u = -\int v_x dy &\implies u_{yy} = -v_{xy}\\
\end{align*}
Since $f$ is smooth, its mixed partials are equal and thus $u_{yy}=u_{xx}$.
\end{solution}
\begin{problem}
  Let's redo \ref{abels-theorem} in the context of complex analysis.
  Consider a sequence $(a_i)$ of complex numbers so that
  $\sum_{i=0}^\infty a_i$ converges to $L$.  Then
  $\lim_{z \to 1^{-}} \sum_{i=0}^\infty a_i z^i = L$ provided
  $z \to 1$ in a \textbf{Stolz sector}, i.e., suppose there is some
  $K$ so that $|1-z| \leq K(1-|z|)$.
\end{problem}
\begin{solution}
\begin{lemma}\label{conditional-row-sum-absolute-column-sum}
Suppose $\sum_i^\infty a_i,$ $\sum_i^\infty b_i=B,$ and $\sum_{i=0}^\infty\sum_{j=0}^\infty f(i,j) = L$ are convergent, with $f(i, j) = \begin{cases}a_ib_j & i \leq j\\0. \end{cases}$.
Then $\sum_{j=0}^\infty\sum_{i=0}^\infty f(i, j)$ converges to L.
\end{lemma}
\begin{proof}
For any $N$, we have the following
\begin{align*}
|\sum_{j=0}^\infty\sum_{i=0}^\infty f(i,j) - \sum_{i=0}^N\sum_{j=0}^\infty f(i,j)| &=
|(\sum_{j=0}^\infty\sum_{i=0}^\infty a_ib_j - \sum_{i=0}^\infty\sum_{j=0}^{i-1} a_ib_j) - (\sum_{i=0}^N a_iB - \sum_{i=0}^N\sum_{j=0}^{i-1}a_ib_j|)\\
&= |\sum_{j=0}^\infty\sum_{i=0}^\infty a_ib_j - B\sum_{i=0}^N a_i - \sum_{i=N+1}^\infty \sum_{j=0}^{i-1} a_ib_j|\\
&\leq |\sum_{j=0}^\infty b_j\sum_{i=0}^\infty a_i - \sum_{j=0}^\infty b_j\sum_{i=0}^N a_i| +  |\sum_{i=N+1}^\infty a_i\sum_{j=0}^{i-1} b_j|\\
&\leq |\sum_{j=0}^\infty \left(b_j\sum_{i=N+1}^\infty a_i\right)| +  |\sum_{i=N+1}^\infty a_i\sum_{j=0}^{i-1} b_j|
\end{align*}
Let $B'$ be the sup of the absolute value of the maximum partial sum of $b_j$. Then, choose $N$ so that $|\sum_{i=N+1}^\infty a_i| < \frac{\epsilon}{2B'}$. Plugging this into the last equation, we can conclude
\begin{align*}
|\sum_{j=0}^\infty\sum_{i=0}^\infty a_ib_j - \sum_{i=0}^N\sum_{j=0}^\infty a_ib_j| &\leq
|B'\frac{\epsilon}{2B'}| +|\frac{\epsilon}{2B'}B'|   \leq \epsilon
\end{align*}
So the two sums are equal.
\end{proof}
With this we move onto the proof of the theorem. First, notice that the condition $|1-z|\leq K(1-|z|)$ implies $|z| <  1$, as if $|z| < 1$ then the RHS is negative (impossible since there is an absolute value). If $|z|=1$ then the RHS may be 0, but the only way the inequality is satisfied is if $z=1$, and that's the limit point.

Thus $\sum a_iz^i$ converges absolutely by comparison to the geometric series $max(a_i)|z|^i$.

We would like to show that the two values have difference zero. Subtract them, expand, and sum `along the columns instead of along the rows'. Lemma \ref{conditional-row-sum-absolute-column-sum} shows that this operation will not mess anything up.
\begin{align*}
\sum a_i - \sum a_iz^i &= \sum a_i(1-z^i)\\
&= \sum_{i=0}^\infty a_i(1-z)(1 + z + \dots + z^{i-1})\\
&= (1 - z)\sum_{j=0}^\infty z^j\sum_{i=j}^\infty a_i
\end{align*}
It suffices to prove that the absolute value of the difference is 0. Note that since $\sum a_i$ converges, $\forall \epsilon > 0$, we can find a number $N$ so that for all $i\geq N$, $L-\sum_{i=1}^{N-1} a_i = \sum_{i=N}^\infty a_i < \epsilon.$
\begin{align*}
|\sum a_i - \sum a_iz^i| &=  \left|(1 - z)\sum_{j=0}^\infty z^j\sum_{i=j}^\infty a_i\right|\\
&\leq \left|K(1 - |z|)\right|\sum_{j=0}^{N-1} |z|^j\left|\sum_{i=j}^\infty a_n\right| + \left|K(1 - |z|)\right|\sum_{j=N}^\infty |z|^j|\epsilon|\\
&\leq \left|K(1 - |z|^{N})(L-\sum_{i=0}^N\sum_{j=0}^N a_i)\right| +
|K\epsilon|\\
&\leq C_1(N)(1 - |z|^{N}) + C_2\epsilon
\end{align*}
For some positive constant $C_2$ and constant function of $N$ $C_1(N)$. We can make the second term go to 0 by choosing $N$ to be huge so that $\epsilon$ vanishes, and we can then independently make the first term go to zero by choosing $z$ close enough to 1 that $1-|z|^N$ is close to 0.
\end{solution}
\section{Prove or Disprove and Salvage if Possible}
As always with PODASIPs, many of statements below are incorrect.  For
full credit, you must not only salvage these false statements, but
also explain why the statement is false (e.g., perhaps by exhibiting a
counterexample).

\begin{problem}
  Suppose $f : \R^2 \to \R^2$ and $g : \R^2 \to \R^2$ are smooth (but
  not necessarily holomorphic when regarded as functions from $\C$ to
  $\C$).  Then by the chain rule,
  \[
    \frac{\partial}{\partial z} \left( f \circ g \right) =
    \frac{\partial f}{\partial z} \frac{\partial g}{\partial z} + \frac{\partial f}{\partial \conj{z}} \frac{\partial g}{\partial \conj{z}}.
  \]
  and similarly
  \[
    \frac{\partial}{\partial \conj{z}} \left( f \circ g \right) =
    \frac{\partial f}{\partial \conj{z}} \frac{\partial g}{\partial \conj{z}} + \frac{\partial f}{\partial z} \frac{\partial g}{\partial z}.
  \]
\end{problem}
\begin{solution}
Surely not, as this would imply that $\pfrac{f\circ g}{z} = \pfrac{f\circ g}{\conj z}$. Plus, $\pfrac{f}{z}$ doesn't make that much sense since $f$ isn't really a function of $z$ but rather $g(z)$.

\begin{align*}
\pfrac{f\circ g}{z} &= \frac{1}{2}(\pfrac{}{x} - i\pfrac{}{y})(f\circ g)
\end{align*}
Define real valued 2 variable functions $u_1, v_1, u_2, v_2$ so that $g(x,y) = u_2(x,y) + iv_2(x,y)$, $f(x, y) = u_1(x, y) + iv_1(x, y)$. Then 
\begin{align*}
(f\circ g)(x+iy) = u_1(u_2(x, y), v_2(x, y)) + iv_1(u_2(x, y), v_2(x, y))
\end{align*}
Applying the chain rule, we can get the $x$ and $y$ derivative:
\begin{align*}
\pfrac{f\circ g}{x} &= \pfrac{f}{u_2}\pfrac{u_2}{x} + i\pfrac{f}{v_2}\pfrac{v_2}{x}\\
\pfrac{f\circ g}{y} &= \pfrac{f}{u_2}\pfrac{u_2}{y} + i\pfrac{f}{v_2}\pfrac{v_2}{y}
\end{align*}
Then we can compute the Wirtinger partial derivative $\pfrac{f\circ g}{z}$
\begin{align*} 
\pfrac{f\circ g}{z} &= \frac{1}{2}(\pfrac{f\circ g}{x} - i \pfrac{f\circ g}{y})\\
&= \frac{1}{2}(\pfrac{f}{u_2}\pfrac{u_2}{x} + i\pfrac{f}{v_2}\pfrac{v_2}{x} -  i\pfrac{f}{u_2}\pfrac{u_2}{y} + \pfrac{f}{v_2}\pfrac{v_2}{y})\\
&= \frac{1}{2}(\pfrac{f}{u_2}(\pfrac{u_2}{x} -i\pfrac{u_2}{y}) + \pfrac{f}{v_2}(i\pfrac{v_2}{x} + \pfrac{v_2}{y}))\\
&= \frac{1}{2}(\pfrac{f}{u_2}(\pfrac{u_2}{x} -i\pfrac{u_2}{y}) - i\pfrac{f}{v_2}(\pfrac{v_2}{x} - i\pfrac{v_2}{y}))\\
&= \pfrac{f}{u_2}\pfrac{u_2}{z} - i\pfrac{f}{v_2}\pfrac{v_2}{z}
\end{align*}
Let $w(x, y) = u_2(x, y) + iv_2(x, y).$ Then $u_2 = \frac{1}{2}(w + \conj{w})$ and $iv_2 = \frac{1}{2}(w - \conj{w}).$

\begin{align*}
\pfrac{f\circ g}{z} &= \frac{1}{2}((\pfrac{f}{w} +\pfrac{f}{\conj{w}})\pfrac{u_2}{z} + (\pfrac{f}{\conj{w}} - \pfrac{f}{w})\pfrac{v_2}{z}\\
&= \frac{1}{2}(\pfrac{f}{w}(\pfrac{u_2}{z} - v\pfrac{v_2}{z}) + \pfrac{f}{\conj{w}}(\pfrac{u_2}{z} + \pfrac{v_2}{z}))\\
&= \pfrac{f}{w}\pfrac{g}{z} + \pfrac{f}{\conj{w}}\pfrac{\bar{g}}{z}
\end{align*}
And by symmetry we can argue that 
\[
\pfrac{f\circ g}{\conj{z}} =\pfrac{f}{\conj{w}}\pfrac{\conj{g}}{\conj{z}} + \pfrac{f}{w}\pfrac{g}{\conj{z}}
\]
\end{solution}


\begin{problem}\label{schwarz-reflection-principle}
If $f : \C \to \C$ is holomorphic, then $z \mapsto \overline{f(\conj z)}$ is holomorphic.
\end{problem}
\begin{solution}
If $f(x+iy) := u_1(x, y) + iv_1(x, y)$ then $ u_2(x, y) + v_2(x, y) := \conj{f}(x - iy) = u_1(x, -y) - iv_1(x, -y).$ Let's check the equations:
\begin{align*}
\pfrac{u_2}{x} &= \pfrac{u}{x} &
\pfrac{v_2}{x} &= -\pfrac{v}{x}\\
\pfrac{u_2}{y} &= -\pfrac{v}{y} & 
\pfrac{v_2}{y} &= \pfrac{v}{y}\\
\end{align*}
Using the fact that Cauchy-Riemann equations are satisfied for $f$,
\begin{align*}
\pfrac{u_2}{x} &= \pfrac{v_2}{y} & 
\pfrac{u_2}{y} &= -\pfrac{v_2}{y}
\end{align*}
So the claim is true.
\end{solution}

\begin{problem}\label{cauchy-riemann-alone-not-sufficient}Suppose $u, v : \mathbb{R}^2 \to \mathbb{R}$ satisfy
  \[
    \frac{\partial u}{\partial x}(0,0)=\frac{\partial v}{\partial y}(0,0)
    \mbox{ and }
    \frac{\partial u}{\partial y}(0,0)=-\frac{\partial v}{\partial x}(0,0).
  \]
  Then $f : \C \to \C$ given by $f(x+iy) = u(x,y) + i \, v(x,y)$ is holomorphic at $0$.
\end{problem}
\begin{solution}
False, define $f(z) = \begin{cases}1&\Re(z) = 0 \text{ or } \Im(z) = 0\\ 0 & \text{otherwise}\end{cases}$. It's not holomorphic at 0 since it's not real differentiable there. On the other hand, if $f$ is real differentiable and satisfies the Cauchy Riemann equations, then it must be complex differentiable.

Choose $z = x + iy\in\C$. Then \[f(z) = (f(x + iy) - f(x)) +  (f(x) - f(0)).\] By the MVT we can find points on the line between $x$ to $x+iy$ and the line between $x$ to $0$ with the correct derivative for the functions $u$ and $v$:
\begin{align*}
u(z) &= xu_x(p_1) + u_y(p_2)y\\
v(z) &= xv_x(p_3) +v_y(p_4)y
\end{align*}
Since the derivatives are all continuous at 0, and the $p_i$ are close to 0, making $z$ closer to zero allows us to find arbitrarily small $\epsilon_i$ so that 
\begin{align*}
u(z) &= (u_x(0)+\epsilon_1)x + (u_y(0)+\epsilon_2)y \\
v(z) &= (v_x(0)+\epsilon_3)x + (v_y(0)+\epsilon_4)y &= (u_y(0)+\epsilon_5)x + (u_x(0)+\epsilon_6)y
\end{align*}
Then we show that the function is complex differentiable by showing the following limit exists:
\begin{align*}
\lim_{z\to 0} \frac{f(z)}{z} &= \lim \frac{(u_x(0)+\epsilon_1)x + (u_y(0)+\epsilon_2)y + i((u_y(0)+\epsilon_5)x + (u_x(0)+\epsilon_6)y)}{x+ iy}\\
&= \lim \frac{u_x(0)(x+iy) + u_y(0)(y-ix) + \epsilon_1x + \epsilon_2y + \epsilon_5x - \epsilon_6y}{x+ iy}\\
&= \lim \frac{u_x(0)(x+iy) + iu_y(0)(x+iy)}{x+iy} + \lim_{z\to 0} \frac{(\epsilon_1+\epsilon_5)x}{x+iy} + \lim_{z\to 0}\frac{(\epsilon_2 - \epsilon_6)y}{x+ iy}\\
&= u_x(0) + iu_y(0)
\end{align*}
Where the second limit disappears because $\epsilon$ goes to zero along with $z$.
\end{solution}

\begin{problem}\label{open-mapping-theorem-preview}
There is a nonconstant holomorphic function with constant absolute value.
\end{problem}
\begin{solution}
False, suppose that $f$ is holomorphic with $|f(z)| = c$. Then $f(x+iy) = u(x, y) + iv(x, y)$ and $u^2 + v^2 = c^2.$ This implies that 
\begin{align}
\label{first_abs_omtp} 2uu_x + 2vv_x &= \frac{d}{dx}c^2 = 0\\
\label{mid_abs_omtp}2uu_y + 2vv_y &= \frac{d}{dx}c^2 = 0\\
\label{second_abs_omtp} 0 &= -2uv_x + 2vu_x \color{purple} \quad\text{ apply CR equations to \ref{mid_abs_omtp}}
\end{align}
Multiplying \ref{first_abs_omtp} by $v$ and \ref{second_abs_omtp} by $u$,
\begin{align*}
2uvu_x + 2v^2v_x &= -2u^2v_x + 2vuu_x\\
2v^2v_x &= -2u^2v_x\\
2(v^2 +u^2) v_x &= 0\\
c^2v_x &= 0
\end{align*}
If $c^2=0$, then $f$ is constant (as $|f(z)|=c=0$ everywhere), and otherwise $v_x$ (and $u_x$ if we do the last step after multiplying \ref{second_abs_omtp} by $-u$ instead) are 0. Using the Cauchy Riemann equations, this implies that all of the derivatives are 0, so $f$ is constant.
\end{solution}
\begin{problem}\label{schwarzian-derivative}For $f : \mathbb{C} \to \mathbb{C}$, the \textbf{Schwarzian derivative} of $f$ is
  \[ (Sf)(z)=\left({\frac{f''(z)}{f'(z)}}\right)'-{\frac12}\left(\frac{f''(z)}{f'(z)}\right)^2={\frac {f'''(z)}{f'(z)}}-{\frac32}\left(\frac{f''(z)}{f'(z)}\right)^2. \]
  If $f$ is a M\"obius transformation, then $(Sf)(z) = 0$.
\end{problem}
\begin{solution}
In the definition, $f$ should be holomorphic as well, so that the notation $f'(z)$ makes sense.

Let $f=\frac{az + b}{cz + d},$ and we can try to compute $Sf$. First let's find $f'$ using the quotient rule:
\begin{align*}
f' &= \frac{(cz + d)a - (az + b)c}{(cz + d)^2}\\
&= \frac{da - bc}{(cz + d)^2}
\end{align*}
When we find the logarithmic derivative, the constant $(da - bc)$ will disappear, making the next computation a bit better:
\begin{align*}
\frac{f''}{f'} &= \frac{-2(cz + d)}{(cz+d)^4} / \frac{1}{(cz+d)^2}\\
&= \frac{-2}{cz + d}
\end{align*}
For the final term:
\begin{align*}
(\frac{f''}{f'})' = \frac{-2}{(cz + d)^2}
\end{align*}
Now we can plug into the equation and quickly see that the value is in fact 0.
\begin{align*}
Sf(z) &= \left({\frac{f''(z)}{f'(z)}}\right)'-{\frac12}\left(\frac{f''(z)}{f'(z)}\right)^2\\
&= \frac{-2}{(cz + d)^2} - \frac{1}{2}(\frac{-2}{cz + d})^2
&= 0
\end{align*}
\end{solution}
\end{document}

